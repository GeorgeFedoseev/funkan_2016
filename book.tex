\documentclass[12pt,a4paper,titlepage,oneside]{book}
\usepackage[utf8]{inputenc}
\usepackage[russian]{babel}
\usepackage[OT1]{fontenc}
\usepackage{amsmath}
\usepackage{amsthm}
\usepackage{mathrsfs}
\usepackage{indentfirst}
\usepackage{amsfonts}
\usepackage{bbold}
\usepackage[hidelinks]{hyperref}
\usepackage{amssymb}
\usepackage[left=2cm,right=2cm,top=2cm,bottom=2cm]{geometry}
\title{Функциональный анализ}

%%% COMMANDS
\newcommand{\overbar}[1]{\mkern 1.5mu\overline{\mkern-1.5mu#1\mkern-1.5mu}\mkern 1.5mu}
\newcommand{\argmax}{\operatornamewithlimits{argmax}}
\newcommand{\argmin}{\operatornamewithlimits{argmin}}

%%% THEOREMS
\newtheoremstyle{break}{3pt}{3pt}{\itshape}{}{\bfseries}{.}{\newline}{}

\theoremstyle{definition}
\newtheorem*{definition}{Определение}

\theoremstyle{plain}
\newtheorem*{theorem}{Теорема}

\theoremstyle{break}
\newtheorem*{theorem-break}{Теорема}

\theoremstyle{remark}
\newtheorem*{remark}{Замечание}

\theoremstyle{remark}
\newtheorem*{example}{Пример}

\theoremstyle{remark}
\newtheorem*{examples}{Примеры}

\theoremstyle{remark}
\newtheorem*{cexample}{Контр-пример}

\theoremstyle{plain}
\newtheorem*{lemma}{Лемма}

\theoremstyle{plain}
\newtheorem*{corollary}{Следствие}

%%% MISC
\setcounter{tocdepth}{1}
\def\labelitemi{--}
\renewcommand{\qedsymbol}{\rule{0.7em}{0.7em}}
\sloppy
\binoppenalty=\maxdimen
\relpenalty=\maxdimen

%%% DOCUMENT
\begin{document}

%%% TITLE PAGE
\begin{titlepage}
\begin{center}

\vfill

Санкт-Петербургский государственный университет\\
\ \\

\vfill

{\large\bf ФУНКЦИОНАЛЬНЫЙ АНАЛИЗ\\}
\ \\
Лекции для студентов факультета ПМ-ПУ\\
(III курс, 6-ой семестр)

\vfill

\hfill\vbox
{
\hbox{Доцент кафедры моделирования электромеханических}
\hbox{и компьютерных систем, кандидат физ.-мат. наук}
\hbox{Владимир Олегович Сергеев}
}

\vfill

Санкт-Петербург, 2016
\end{center}
\end{titlepage}

\tableofcontents

\vfill

\begin{small}
При составлении односеместрового курса лекций по функциональному анализу выбраны направления, связанные с постановкой задач прикладной математики. Эти направления представлены в следующих уже классических работах:
\begin{enumerate}
    \item Л.~В.~Канторович и Г.~П.~Акилов. Функциональный анализ в нормированных пространствах, Физматгиз, 1959.
    \item В.~А.~Треногин. Функциональный анализ. Москва <<Наука>>\, главная редакция физ.-мат. литературы, 1980.
    \item В.~А.~Садовничий. Теория операторов. Издательство Московского университета, 1986.
\end{enumerate}
В издании текста лекций существенную помощь оказали студенты 3-го курса (2016 г.): \\
307 гр. --- Авдеева Анастасия, Улитина Ирина;\\
308 гр. --- Малых Егор, Огурцова Анастасия;\\
310 гр. --- Арзуманян Наринэ, Тагирова Рената, Тищук Максим.
\end{small}

\chapter{Нормированные пространства}

Изучение свойств отображений, как и в математическом анализе, начнём с введения определений, связанных с областями задания отображений.

\section{Метрические пространства}

\begin{definition}
В метрическом пространстве $X$ для любых элементов $x, y \in X$ определено расстояние $\rho(x, y)$, которое удовлетворяет требованиям (аксиомам метрического пространства):
\begin{enumerate}

	\item $\rho(x, y)\geqslant 0$, и $\rho(x, y)=0$ означает, что элементы $x$ и $y$ совпадают,

	\item $\rho(x, y)=\rho(y, x)$,

	\item $ \rho(x, y)\leqslant \rho(x, z) +\rho(z, y)$ --- неравенство треугольника.
	
\end{enumerate}
\end{definition}

Расстояние (метрика) $\rho$ определяет сходимость последовательности $\lbrace x_n \rbrace_{n=1}^{\infty} \in X$ к элементу $x^{*} \in X$:

\begin{center}

$x_n\to x^*$, если $\rho(x_n, x^{*})\to 0$ при $n\to\infty$

\end{center}

Из аксиом метрического пространства следует непрерывность функции $\rho(x, y)$, то есть если $x_n\to x^{*}$, $y_n\to y^{*}$ при $n \to \infty$, то $\rho(x_n, y_n) \to \rho(x^*, y^*)$.

Естественным образом вводятся понятия:

\begin{itemize}

	\item  открытый шар, замкнутый шар, окрестность элемента $x_0\in X$,

	\item внутренняя точка множества $M \in X$, открытое множество, замкнутое множество,

	\item подпространство $X_0$ метрического пространства: метрика в $X_0$ определяется метрикой пространства $X$, множество $X_0$ --- замкнуто,

	\item фундаментальная последовательность $\lbrace x_n \rbrace_{n=1}^{\infty}$ --- последовательность, такая что $\forall \varepsilon > 0$ существует номер $n = n(\varepsilon)$ такой что $\rho(x_n, x_{n+m}) < \varepsilon$ при $n > n(\varepsilon)$, $m \geqslant 1$ (последовательность, сходящаяся в себе).

\end{itemize}

\subsection*{Основные типы метрических пространств и множеств}

\begin{enumerate}

	\item \textbf{Полное метрическое пространство} --- любая фундаментальная последовательность имеет предел, принадлежащий $X$ (в математическом анализе --- признак Коши сходимости числовой последовательности).

	\item Множество $D \in X$ \textbf{плотно в множестве} $ M_0 \in X$, если для каждого элемента $x_0 \in M_0$ и любого $\varepsilon > 0$ найдётся элемент $z \in D$, такой что $\rho(x_0, z) < \varepsilon$($z = z(\varepsilon)$). Если множество $D$ плотно в $M_0$, то для любого элемента $x_0 \in  M_0$ существует последовательность элементов $\lbrace z_n \rbrace \in D$ таких, что $\rho(z_n, x_0) \to 0$ при $n \to \infty$. Ясно, что $\overbar{D} = M_0$.

	\item \textbf{Сепарабельное пространство $X$} --- в таком пространстве существует счётное всюду плотное множество $D$: $D=\lbrace x_1,x_2,\ldots,x_n,\ldots \rbrace$. Для любого элемента $x_0 \in X$ можно найти такой номер $n = n(x_0,\varepsilon)$, что $\rho(x_0, z_n) < \varepsilon$.

	\begin{example}	
	Пространство $C[a,b]$ непрерывных на $[a,b]$ функций сепарабельно. В математическом анализе это теорема Вейерштрасса: для каждой непрерывной функции $x_0(t)$ существует полином $P_n$ с рациональными коэффициентами такой что
	
	\begin{equation*}
	\rho(x_0, P_n)=\smash{\displaystyle\max_{t \in [a,b]}} |x_0(t)-P_n(t)| < \varepsilon
	\end{equation*}

	Множество таких полиномов счётно.
	\end{example}

	\item \textbf{Компактное множество метрического пространства $X$}

	Множество $K$ компактно в $X$, если в любой подпоследовательности элементов $\lbrace x_n \rbrace_{n=1}^{\infty} \subset K$ существует фундаментальная подпоследовательность $\lbrace x_{n_k} \rbrace$, $n_{k+1}>n_k$ (то есть последовательность $n_k$ возрастает).

\end{enumerate}


Компактность множеств играет важную роль при исследовании приближённых методов решения задач.

Если $X$ полное пространство, то существует предельный элемент этой фундаментальной подпоследовательности, но он может не принадлежать множеству $K$.

Определение компактного множества не конструктивно. Следующая теорема Хаусдорфа даёт эффективный критерий компактности множеств.

\begin{definition}
Говорят, что в множестве $M \in X$ существует конечная $\varepsilon$-сеть $\lbrace x_1,x_2,\ldots,x_{N(\varepsilon)} \rbrace$, если для любого элемента $x \in M$ можно указать элемент $x_n$ $\varepsilon$-сети, такой что

$$
\rho(x_n, x) < \varepsilon \mbox{, } n=n(x,\varepsilon).
$$

\end{definition}

\begin{theorem}[Хаусдорф, около 1914 г.]
Для того, чтобы множество $K \subset X$ было компактно в $X$, необходимо и достаточно чтобы для любого $\varepsilon > 0$ в множестве $K$ существовала конечная $\varepsilon$-сеть. 
\end{theorem}
\begin{proof}

$\underbar{Необходимость}$ (от противного). Пусть $K$  --- компактное в $X$ множество. Предположим, что для заданного $\varepsilon >0 $ не существует конечной $\varepsilon$-сети. Возьмем любой элемент $x_1 \in K.$ Согласно предположению он не образует конечной $\varepsilon$-сети и существует элемент $x_2 \in K$ такой что $\rho(x_1, x_2) > \varepsilon.$ Два элемента $x_1$ и $x_2$ не образуют $\varepsilon$-сети, и существует третий элемент $x_3 \in K$ такой что значения $\rho(x_3, x_1), \; \rho(x_3, x_2), \; \rho(x_2, x_1) > \varepsilon.$

Продолжая этот процесс, получим последовательность элементов $x_1,x_2,x_3, \ldots, x_n, \ldots \in K$ таких что $\rho(x_i, x_j) > \varepsilon$ при   $i \neq j$. Из этой последовательности нельзя составить ни одной фундаментальной подпоследовательности, что противоречит компактности множества $K.$

$\underbar{Достаточность}$. Пусть $\{x_n\}_{n=1}^{\infty}$ любая последовательность элементов множества $K.$ Образуем последовательность чисел $\varepsilon _k > 0,$ монотонно стремящуюся к $0.$

Для значения $\varepsilon _1$  в множестве $K$ существует конечная $\varepsilon _1$-сеть, т.е. все множество $K$ может быть покрыто конечным числом шаров радиуса $\varepsilon _1 .$ Так как последовательность $\{ x_n \}$ содержит бесконечное число элементов, то среди упомянутых шаров найдется хотя бы один шар $V_{\varepsilon _1} (z_1),$ в котором содержится бесконечное число элементов последовательности $\{x_n\}.$ Обозначим $x_{n_1}$ первый из таких элементов: $x_{n_1} \in V_{\varepsilon _1} (z_1).$

Далее, шар $ V_{\varepsilon _1} (z_1) $ может быть покрыт конечным числом шаров радиуса $\varepsilon _2 < \varepsilon _1.$ Тогда в одном из таких шаров $V_{\varepsilon _2} (z_2)$ содержится бесконечное число элементов последовательности $\{x_n\},$ и первый после $x_{n_1}$ такой элемент обозначим $x_{n_2}:$
\begin{equation*}
x_{n_1} \in V_{\varepsilon _1} (z_1) \cap V_{\varepsilon _2} (z_2),
\end{equation*}
Продолжая этот процесс, получим последовательность элементов $\{x_{n_k}\} \subset \{x_n\}$ таких что $x_{n_k} \in V_{\varepsilon _k} (z_k),$
\begin{equation*}
x_{n_k} \in \bigcap\limits_{i=1}^{k} V_{\varepsilon _i} (z_i),
\end{equation*}
где $n_k$  возрастающая последовательность чисел. При $m > k$ оба элемента $x_{n_k}$ и $x_{n_m}$ принадлежат шару $V_{\varepsilon _k} (z_k).$ По неравенству треугольника 
\begin{equation*}
\rho(x_{n_k}, x_{n_m}) \leqslant \rho(x_{n_k}, z_k) + \rho(z_k, x_{n_m}) \leqslant \varepsilon_k + \varepsilon_k = 2 \varepsilon _k.
\end{equation*}
 
Следовательно подпоследовательность $\{x_{n_k}\}$ последовательности $\{x_n\}$ является фундаментальной.
\end{proof}

\begin{corollary}
Если в множестве $K \subset X$ существует компактная в $X$ $\varepsilon$-сеть $H_{\varepsilon},$ то множество $K$ компактно в $X.$ 
\end{corollary}

Действительно, так как $H_{\varepsilon} $ является $\varepsilon$-сетью для множества $K,$ то для любого элемента $x \in K$ существует элемент $x_{\varepsilon} \in H_{\varepsilon}$ такой что $\rho(x, x_{\varepsilon}) < \varepsilon.$ Из условия компактности множества $H_{\varepsilon}$ в пространстве $X$ следует, что в  $H_{\varepsilon}$ существует конечная $\varepsilon$-сеть элементов $\bar{x}_1, \bar{x}_2,\ldots \bar{x}_n,$ и для элемента $x_{\varepsilon}$ существует элемент $\bar{x}_k \in H_{\varepsilon}$ такой что $\rho(x_{\varepsilon}, \bar{x}_k) < \varepsilon.$ Тогда $\rho(x, \bar{x}_k) \leqslant \rho(x, x_{\varepsilon}) + \rho(x_{\varepsilon}, \bar{x}_k) \leqslant \varepsilon + \varepsilon = 2 \varepsilon.$

Следовательно элементы $\bar{x}_1, \bar{x}_2,\ldots \bar{x}_n,$ множества $H_{\varepsilon}$ образуют в множестве $K$ конечную $2 \varepsilon$-сеть. По теореме Хаусдорфа множество $K$ компактно в $X.$

Ясно, что компактное множество ограничено: существует шар конечного радиуса, которому принадлежит компактное множество.

\subsection*{Примеры}
\begin{enumerate}

	\item Любое ограниченное множество в конечномерном пространстве компактно (в математическом анализе это теорема Больцано-Вейерштрасса).

	\item Множества, компактные в пространстве непрерывных функций.

\begin{theorem} [Чезаро Арцела, 1870; Джулио Асколи, 1900]
Для того, чтобы множество $E \in C[a, b]$ было компактным, необходимо и достаточно выполнения двух условий:

\begin{itemize}

	\item  Все функции $x\in E$ ограничены в совокупности: $\smash{\displaystyle\max_{ [a,b]}} |x| < const$.

	\item  Все функции $x\in E$ равностепенно непрерывны: $\forall \varepsilon>0$ существует такое число $\delta>0$, что для любых значений $t^{'}, t^{''} \in [a, b]$ таких, что $|t^{''}-t^{'}| <\delta$ верно неравенство $|x(t^{''})-x(t^{'})| <\varepsilon$, где $\delta$ не зависит от выбора функции $x$ из $E$.
	
\end{itemize}

\end{theorem}
\begin{proof} 
	$\underbar{Достаточность}$ Пусть известно, что для функции $x\in E$ выполнены условия теоремы. Построим в $E$ компактную 	$\varepsilon$-сеть $H_{\varepsilon}$. Для 	$ \varepsilon>0$ найдем значение $\delta>0$ такое, что $|x(t^{''})-x(t^{'})| <\varepsilon$ для всех $t^{'}, t^{''}$ таких, что $|t^{''}-t^{'}| <\delta$. Построим конечное число узлов $\{t_i\}$, $a=t_1<t_2<\ldots<t_n=b, t_{k+1}-t_k< \delta$ и зафиксируем их. Рассмотрим множество ломаных $\overbar{x}(t)$ с вершинами в точках $(t_k, \eta_k )$, где $\eta_k=x(t_k)$. Множество всех таких ломаных, построенных для функций множества $E$, обозначим $H_{\varepsilon}$.
	
	Множество $H_{\varepsilon}$ компактно в $C[a, b]$. Действительно, каждая ломаная определяется $n$ числами $(\eta_1, \eta_2,\ldots,\eta_n )$, где все числа $\eta_k$ ограничены: $|\eta_k|\leqslant const$. Из любой последовательности ломаных из $H_{\varepsilon}$ можно образовать фундаментальную последовательность.
	
	Покажем, что компактное множество $H_{\varepsilon}$ образует в $E$ $\varepsilon$-сеть. Для любой функции $x\in E$ построим ломаную $\overbar{x}(t)$, $\overbar{x} \in H_{\varepsilon}$. Так как $x(t)$ непрерывна, то на отрезке $[t_k, t_{k+1}]$ она достигает своего максимального значения $M_k$ и своего минимального значения $m_k$: $m_k \leqslant x(t) \leqslant M_k, t \in [t_k, t_{k+1}]$. В этих же пределах лежат и значения линейной функции $\overbar{x}(t)$. Ясно, что $|x(t)-\overbar{x}(t)|\leqslant M_k-m_k, t \in [t_k, t_{k+1}]$.
	
	В силу выбора значения $\delta$ величины $M_k-m_k < \varepsilon$. Тогда и $\rho(x, \overbar{x}) < \varepsilon$. Согласно следствию теоремы Хаусдорфа, множество $E$ компактно в $C[a, b]$.
	
	$\underbar {Необходимость}$ Свойства функций из компактного множества $E$, указанные в теореме, сразу следуют из существования в $E$ $\underbar {конечной}$ $\varepsilon$-сети непрерывных на $[a, b]$ функций $x_1(t), x_2(t),\ldots,x_N(t)$.
\end{proof}

	\item Множества, компактные в пространстве суммируемых со степенью $p, (p \geqslant 1)$ функций.

\begin{theorem} [Марсель Рисс, 1935 г.]
Для того, чтобы множество $E$  пространства $L_p[a, b]$ было компактным, необходимо и достаточно, чтобы

\begin{itemize}

	\item $\int\limits_a^b |x(t)|^p dt \leqslant const, x \in E$,

	\item при $\tau \to 0$ интегралы $\int\limits_a^b |x(t+\tau)-x(t)|^p dt \to 0$ равномерно относительно $x \in E$.

\end{itemize}

\end{theorem}

Условия теоремы М. Рисса аналогичны условиям теоремы Арцела-Асколи для пространства непрерывных функций. Доказательство теоремы основано на плотности множества непрерывных на $[a, b]$ функций в пространстве $L_p [a, b]$.

	\item Принцип вложенных шаров (В математическом анализе --- лемма о вложенных отрезках).

\begin{theorem} Пусть в полном метрическом пространстве $X$ дана последовательность замкнутых шаров $\overbar{V}_{r_n}(x_n)$:

\begin{equation*}
\overbar{V}_{r_{n+1}}(x_{n+1})\subset \overbar{V}_{r_{n}}(x_n) \mbox{, где } r_n \to 0 \mbox{ при } n \to \infty
\end{equation*}

Тогда в $X$ существует и единственен элемент $x^{*}$, принадлежащий всем шарам $\overbar{V}_{r_{n}}(x_n)$

\end{theorem}

\begin{proof} Для расстояний $\rho(x_n, x_{n+m})$ между центрами этих шаров верно 

\begin{equation*}
\rho(x_n, x_{n+m})<r_n \to 0 \mbox{ при } n \to \infty
\end{equation*}

и последовательность $\{x_n\}$ --- фундаментальная последовательность. Так как пространство $X$ полное, то существует $\displaystyle\lim_{n\to \infty} x_n=x^{*} \subset X$ и этот предел единственен.

Ясно, что $x^{*}$ принадлежит всем шарам $\overbar{V}_{r_{n}}(x_n)$.
\end{proof}

\end{enumerate}

\section{Пространства первой и второй категории}

Множество $D$ всюду плотно в метрическом пространстве $X$, если для каждого элемента $x \in X$ и любого $\varepsilon > 0$ существует элемент $z \in D$ такой что
$\rho(x,z) < \varepsilon$.

Сформулируем утверждение: множество $E$ не является множеством всюду плотным в пространстве $X$.

\begin{definition}
Множество $E$ \textbf{не является всюду плотным} в $X$ (\textbf{нигде не плотным} в $X$), если в любом замкнутом шаре $\overbar{V}_r(x)$ существует замкнутый шар, в котором нет элементов множества $E$.
\end{definition}

\begin{example}
На плоскости (в пространстве $R_2$) множество точек любой прямой --- нигде не плотное в $R_2$ множество.
Рассмотрим множество прямых $l$, параллельных оси $x$ и пересекающих ось $y$ в точках с рациональными значениями координат $y_n$. Ясно, что множество таких прямых счётно, и каждая прямая $l_n$ этого множества есть множество нигде не плотное в пространстве $R_2$.
\end{example}

Будет ли множество точек $\bigcup\limits_{k=1}^\infty l_k$ совпадать со всем пространством $R_2$? Ответ отрицателен: прямые, пересекающие ось $y$ в точках с иррациональными значениями, не принадлежат $\bigcup\limits_{k=1}^\infty l_k$.

\begin{definition}
Множество $E$ называется множеством первой категории, если оно представимо в виде 
$$E=\bigcup\limits_{k=1}^\infty E_k,$$ 
где все множества $E_k$ нигде не плотные в $X$. Если множество $E$ нельзя представить в виде счетного объединения нигде не плотных множеств, то $E$ называется множеством второй категории.
\end{definition}

В общем случае верна теорема:

\begin{theorem}[Луи Бэр, 1905 г.]
Полное метрическое пространство $X$ является множеством второй категории.
\end{theorem}

\begin{proof}
(От противного)

Предположим, что $X=\bigcup\limits_{k=1}^\infty E_k$, где все множества нигде не плотные. Пусть $\overbar{V}_{r_0}(x_0)$ произвольный шар. Так как $E_1$ нигде не плотно в $X$, то в этом шаре существует шар $\overbar{V}_{r_1}(x_1)$, в котором нет элементов множества $E_1$. Можно считать, что радиус этого шара $r_1 < \frac{1}{2}r_0$.

Множество $E_2$ нигде не плотно: в шаре $\overbar{V}_{r_1}(x_1)$ существует шар $\overbar{V}_{r_2}(x_2)$, в котором нет элементов множества $E_2$ (и элементов множества $E_1$). Можно считать, что $r_2 < \frac{1}{2}r_1 < \frac{1}{2^2}r_0$.

Продолжая этот процесс, получим последовательность вложенных шаров 
$$\overbar{V}_{r_1}(x_1) \supset \overbar{V}_{r_2}(x_2) \supset \ldots \supset \overbar{V}_{r_n}(x_n) \supset \ldots$$
$$r_n < \frac{1}{2^n}r_0, r_n \to 0$$
при
$n \to \infty$, и в каждом шаре $\overbar{V}_{r_n}(x_n)$ нет элементов множеств $E_1, E_2, \ldots, E_n$.

По теореме о вложенных шарах существует элемент $x^*\subset X$:
$x_n \to x^*$ при $n \to \infty$.

Ясно, что $x^*\in \bigcup\limits_{n=1}^\infty \overbar{V}_{r_n}(x_n)$ и следовательно $x^*\notin \bigcup\limits_{n=1}^\infty E_n$, что противоречит предположению $X=\bigcup\limits_{n=1}^\infty E_n$. Остается принять, что множество $X$ пространство второй категории.
\end{proof}

\section{Линейные пространства, нормированные пространства, пространства Банаха}

\begin{definition}

Множество элементов $X$ называется \textbf{линейным множеством}, если для его элементов определены действия сложения и умножения на число (вещественное или комплексное), не выводящие из множества $X$:

\begin{itemize}

	\item если $x,y\in X$, то $x+y\in X$

	\item если $x\in X$, то $\lambda x\in X$

\end{itemize}

\end{definition}


Эти действия должны удовлетворять обычным условиям (аксиомам). Если $\lambda$ вещественные числа, то $X$ --- вещественное линейное множество, если $\lambda$ комплексные, то $X$ --- комплексное линейное множество. Для вещественного линейного множества $X$ можно построить комплексное линейное множество $Z$: достаточно ввести элементы $z=x+iy,\quad x,y \in X$ и определить сумму элементов:

\begin{equation*}
z_1+z_2=(x_1+x_2)+i(y_1+y_2)
\end{equation*}
и ввести умножение на комплексное число $\lambda$:
\begin{equation*}
\lambda z = (\alpha + i\beta)(x+iy) = (\alpha x - \beta y)+i(\alpha y + \beta x)
\end{equation*}
(комплексификация линейного множества $X$).

Для комплексного линейного множества $Z$ каждый элемент $z=x+iy$, где $x$ и $y$ --- элементы вещественного множества.

Рассмотрим вещественное пространство $X$ пар $(x,y)$, в котором определим сумму: $(x_1,y_1)+(x_2,y_2) = (x_1+x_2,y_1+y_2)$ и умножение на вещественное число $\lambda$: $\lambda(x,y) = (\lambda x, \lambda y)$. Множество пар $(x,y)$ образует вещественное линейное множество  $X$ (декомплексификация комплексного линейного множества $Z$).

Из аксиом линейного множества отметим некоторые следствия:

\begin{enumerate}

	\item Существования нулевого элемента: $\mathbb{0}=x-x=(1-1)x=0x$.

	\item Из равенства $\lambda x=0$ при $\lambda\ne0$ следует $x=\mathbb{0}$.

	\item Определение линейной независимости элементов $\{x_1,x_2, \ldots, x_n\}$.

	\item Определение размерности линейного множества $X$ как наибольшего числа линейно независимых элементов множества $X$.

	\item Линейное множество бесконечномерно, если для любого натурального $n$ существует $n$ линейно независимых элементов.

\end{enumerate}

\subsection*{Примеры линейных множеств}

\begin{enumerate}

	\item Вещественное пространство $V_n$ $n$-мерных векторов.

	\item Множество прямоугольных матриц размерности $(n\times m)$.

	\item Множество $C[t_0,t_1]$ непрерывных на $[t_0,t_1]$ функций $x$. Функции $x_k(t)=t^k$, $k=1,2,\ldots$ линейно независимы, а пространство $C[t_0,t_k]$ бесконечномерно.

	\item Множество решений $x\in C^n[t_0,t_1]$ уравнения

	\begin{equation*}
	\frac{d^nx(t)}{dt^n}+a_1\frac{d^{n-1}x(t)}{dt^{n-1}}+\ldots+a_{n-1}\frac{dx(t)}{dt}+a_n=0 \mbox{, где } a_k\in C[t_0,t_1]
	\end{equation*}

\end{enumerate}


Снабжая линейное пространство метрикой, мы получаем более богатую теорию. Связь метрики с алгебраическими действиями реализуется введением норм элементов $x$: норма $\lVert x\rVert$ элемента $x\in X$, согласно определению есть число, которое должно удовлетворять трем условиям:

\begin{enumerate}

	\item $\lVert x\rVert \geqslant 0$; если $\lVert x\rVert = 0$, то $x=\mathbb{0}$.

	\item $\lVert \lambda x \rVert = \lvert \lambda \rvert \lVert x \rVert$.

	\item $\lVert x+y \rVert\leq\lVert x \rVert + \lVert y \rVert$.

\end{enumerate}

Норма $\lVert x\rVert$ является непрерывной функцией: $\lvert\lVert x+\triangle x\rVert-\lVert x\rVert\rvert\to0$ при $\lVert\triangle x\rVert\to0$.

Верно неравенство $\lVert x-y\rVert \geqslant \lvert\lVert x\rVert-\lVert y\rVert\rvert$.

Определим метрику в линейном пространстве $X$: $\rho(x,y)=\lVert x-y\rVert$. Ясно, что введенная таким образом метрика удовлетворяет всем аксиомам метрического пространства. Линейное множество с метрикой, определяемой нормой элементов, называется \textbf{нормированным пространством}. Если нормированное пространство полное, то оно называется \textbf{пространством Банаха} (Стефан Банах, 1892-1945, польский математик), банаховым пространством, В-пространством.

\begin{definition}
\textbf{Подпространством нормированного пространства $X$} называется любое линейное \underline{замкнутое} множество $X_0 \in X$.
\end{definition}

\begin{examples}
\leavevmode
\begin{enumerate}

	\item Банаховы пространства $n$-мерных векторов получаем введением различных норм векторов $\bar{x}(x_1,x_2,\ldots,x_n)$:

	\begin{itemize}

		\item $\lVert \bar{x}\rVert_\infty=\displaystyle\max_{i}\lvert x_i\rvert$,

		\item $\lVert \bar{x}\rVert_1 = \displaystyle\sum_{i}\lvert x_i\rvert$,

		\item $\lVert \bar{x}\rVert_2 = \Big( \displaystyle\sum_{i}\lvert x_i\rvert^2 \Big)^{\frac{1}{2}}$

	\end{itemize}

	\item Бесконечномерное банахово пространство $C[t_0,t_1]$ функций $x(t)$ непрерывных на $[t_0,t_1]$. Норма:

	\begin{equation*}
	\lVert x\rVert = \displaystyle\max_{t}\lvert x(t)\rvert
	\end{equation*}

	функции $x_k(t)=t^k, k=1,2,3,\ldots$ линейно независимы.

	\item Бесконечномерное банахово пространство $C_n[t_0,t_1]$. Норма:

	\begin{equation*}
	\lVert x\rVert = \displaystyle\sum_{k=0}^n\max_{i}\lvert\frac{d^kx(t)}{dt^k}\rvert
	\end{equation*}

	\item Пространство Банаха $L_p(a,b)$ измеримых и суммируемых со степенью $p,\;p \geqslant 1,$ функций. Норма:

	\begin{equation*}
	\lVert x\rVert^p = \Big(\int\limits_a^b\lvert x(t)\rvert^p dt \Big)^{\frac{1}{p}}
	\end{equation*}

	Множество полиномов с рациональными коэффициентами плотно в этих пространствах.

	\item Пример неполного нормированного пространства.

	В линейном множестве $C[0,1]$ непрерывных функций введем норму (и метрику):

	\begin{equation*}
	\lVert x\rVert = \Big(\int\limits_a^b\lvert x(t)\rvert^pdt \Big)^{\frac{1}{p}},\quad\rho(x,y) = \Big(\int\limits_0^1\lvert x(t)-y(t)\rvert^pdt \Big)^{\frac{1}{p}}
	\end{equation*}

	Получаемое пространство не является полным. Действительно, последовательность функций $x_k(t)=t^k$ является фундаментальной последовательностью:

	\begin{equation*}
	\lVert x_{n+m}-x_n\rVert^p=\int\limits_0^1(t^n-t^{n+m})^pdt=\int\limits_0^1t^{np}(1-t^m)^pdt<\int\limits_0^1t^{np}dt=\frac{1}{np+1}\to0 \mbox{, при } n \to\infty
	\end{equation*}

	Предел же $\lim x_n(t)$ при $n\to\infty$ в пространстве $C[0,1]$ не существует.

\end{enumerate}

\end{examples}

\section{Пространства Гильберта}

Рассматривается линейное комплексное пространство, в котором введено скалярное произведение $(x, y)$ элементов $x$ и $y$, удовлетворяющее обычным свойствам скалярного произведения:

\begin{enumerate}

    \item $(x, y) = \overbar{(y, x)}$

    \item $(\lambda x_1 + \mu x_2, y) = \lambda (x_1, y) + \mu (x_2, y)$

    \item $(x, x)$ --- вещественное число, $(x, x) \geqslant 0$ и если $(x, x) = 0$, то $x = \mathbb{0}$

\end{enumerate}

Верно неравенство Коши-Буняковского:

\begin{equation*}
\lvert (x, y) \rvert ^2 \leqslant (x, x) \cdot (y, y)
\end{equation*}

Действительно:
\begin{align*}
(x + \lambda y, x + \lambda y) &\geqslant 0,\\
(x, x) + \lambda (y, x) + (x, \lambda y) + \lambda \overbar{\lambda} (y, y) &> 0,\\
(x,x) + 2 Re(\lambda y, x) + \lvert \lambda \rvert^2 (y, y) &\geqslant 0\\
\end{align*}
для любых чисел $\lambda$.
Если $(y, y) = 0$, $y = \mathbb{0}$, то доказываемое утверждение верно. Если $(y, y) \neq 0$, то положим 
\begin{align*}
\lambda &= - \frac{(x, y)}{(y, y)},\\
(\lambda y, x) &= - \frac{ \lvert (x, y)\rvert^2}{(y, y)}.
\end{align*}
Тогда
\begin{align*}
(x, x) - 2 \frac{ \lvert (x, y) \rvert^2}{(y, y)} + \frac{\lvert (x, y) \rvert^2}{(y, y)} &\geqslant 0,\\
(x, x) - \frac{\lvert (x, y) \rvert^2}{(y, y)} &\geqslant 0
\end{align*}
и
\begin{align*}
(x, x) (y, y) &\geqslant \lvert (x, y) \rvert^2,\\
\lvert (x, y) \rvert^2 &\leqslant (x, x) (y, y).
\end{align*}

Введём норму элементов $\lVert x \rVert = \sqrt{(x, x)}$ и получим нормированное пространство. Действительно

\begin{itemize}

	\item $\lVert \lambda x \rVert^2 = (\lambda x, \lambda x) = \lvert \lambda \rvert^2 (x, x) = \lvert \lambda \rvert^2 \lVert x \rVert^2$ и $\lVert \lambda x \rVert = \lvert \lambda \rvert \lVert x \rVert$
	
	\item $\Vert x + y \Vert^2 = (x + y, x + y) = (x, x) + 2 \lvert (x, y) \rvert + (y, y) \leqslant$ (по неравенству Коши-Буняковского) $\leqslant \Vert x \Vert^2 + 2 \Vert x \Vert \Vert y \Vert + \Vert y \Vert^2 = (\Vert x \Vert + \Vert y \Vert)^2$ т.е. и третье условие в определении нормы тоже выполнено: $\Vert x + y \Vert \leqslant \Vert x \Vert + \Vert y \Vert$.

\end{itemize}

Если полученное нормированное пространство \underline{полно}, то оно называется \textbf{пространством Гильберта} (Давид Гильберт, 1862-1943, немецкий математик).

\subsection*{Примеры гильбертовых пространств}

\begin{enumerate}
	\item Пространство $l^2$ числовых последовательностей $x = \lbrace \xi_k \rbrace_{k=1}^{\infty}$ таких, что ряд $\displaystyle\sum_{k=1}^{\infty} \lvert \xi_k \rvert^2$ сходится. Скалярное произведение $(x, y) = \displaystyle\sum_{k=1}^{\infty} \xi_k \overbar{\eta}_k$ и норма $\Vert x \Vert^2 = \displaystyle\sum_{k=1}^{\infty} \lvert \xi_k \rvert^2$.
	
	\item Пространство $L_{\varphi}^2 (a, b)$.
	
	$$ (x, y) = \displaystyle\int\limits_a^b \varphi(t) x(t) \overbar{y}(t) dt$$
	
	$$ \Vert x \Vert^2 = \displaystyle\int\limits_a^b \varphi(t) \lvert x(t) \rvert^2 dt$$
	
	Функция $\varphi(t)$ --- функция, суммируемая на $(a, b)$ такая, что $\varphi(t) > 0$ почти везде.
	
	Пространство $L_{\varphi}^2 (a, b)$ сепарабельно: проведя декомплексификацию этого пространства, получим вещественные пространства, в которых множества полиномов с рациональными коэффициентами являются всюду плотными.
\end{enumerate}

Введение скалярного произведения определяет понятия:

\begin{itemize}
	\item ортогональных элементов: если $(x, y) = 0$, то пишут $x \perp y$
	
	\item ортогональных множеств
\end{itemize}

\begin{theorem}
	Если $H_1$ подпространство пространства $H$, а $H_2$ --- ортогональное дополнение $H_1$, то любой элемент $x \in H$ можно представить единственным образом в виде $x = x_1 + x_2$, где $x_1 \in H_1$, а $x_2 \in H_2$ и $\rho(x, H_1) = \Vert x - x_1 \Vert$
\end{theorem}

\begin{theorem}
	Если $H$ --- сепарабельное гильбертово пространство, то в нём существует не более чем счётная ортогональная система элементов $\lbrace \varphi_1, \varphi_2, \ldots, \varphi_n, \ldots \rbrace$ (базис пространства $H$), и любой элемент $x \in H$ представим в виде 
	$$x = \displaystyle\sum_{k=1}^{\infty} c_k \varphi_k$$
	где $c_k = (x, \varphi_k)$ и $\displaystyle\sum_{k} \lvert c_k \rvert^2 = \Vert x \Vert^2$ ($c_k$ - коэффициенты Фурье элемента $x$ в базисе $\varphi_k$).
\end{theorem}

\begin{examples}
\leavevmode
\begin{enumerate}
	\item Функции $1, \cos \frac{2 \pi (t - a)}{b - a}, \sin \frac{2 \pi (t - a)}{b - a}, \ldots, \cos \frac{2 \pi k (t - a)}{b - a}, \sin \frac{2 \pi k (t - a)}{b - a}$ образуют базис вещественного сепарабельного гильбертова пространства $L_2(a, b)$
	
	\item Полиномы Лежандра $P_k(t)$ степени $k$ ($k = 0, 1, 2, \ldots$) образуют базис вещественного сепарабельного гильбертова пространства $L_2(a, b)$
	
	\item Полиномы Чебышева $T_k(t)$ образуют базис пространства $L_{\varphi}^2 (a, b)$, где $\varphi(t) = \frac{1}{\sqrt{(1 - t^2)}}$
\end{enumerate}
\end{examples}

\chapter{Линейные операторы в нормированных пространствах}

\section{Линейные операторы в нормированных пространствах}
Пусть $X$ и $Y$ два множества и множество $D \subset X$. Если каждому элементу $x \in D$ поставлен в соответствие элемент $y \in Y$, то говорят, что задано отображение $F$ с областью задания $D=D(F)$. Множество элементов $y \in Y$, таких что $y=F(x)$, где $x \in D$, называется областью значений отображения $F$. Естественным образом вводятся понятия обратного отображения ${F}^{-1}$ и взаимно-однозначного отображения. Для метрических пространств $X$ и $Y$ рассматривается непрерывность отображения на элементе $x_0 \in D$ и непрерывность отображения $D$ на множестве $X_0 \subset D$.

Полезным свойством отображения является \textbf{замкнутость} отображения.

\begin{definition} Отображение $F$ замкнуто, если для любой последовательности  $\lbrace x_n \rbrace$

\begin{enumerate}

 \item имеющей предел $\lim x_n= x^* \in D$ при $n\to\infty$,
 
 \item и такой, что существует $\lim F(x_n)= y^* \in Y$
 
\end{enumerate} 
 
верно равенство $F(x^*)=y^*$.

\end{definition}

\begin{example}
$D=[-1;1]$, $Y=[0;\infty)$
\begin{equation*}
F(x) = 
\begin{cases}
   1+x &\text{, если $-1\leqslant x \leqslant 0$}\\
   x^{-1} &\text{, если $0< x \leqslant 1$}
\end{cases}
\end{equation*}

Ясно, что это отображение не является непрерывным на $[-1;1]$, единственная точка разрыва $x=0$. Но это отображение замкнуто:

\begin{enumerate}

 \item при $x_n \to -0$ существует $\lim x_n= x^*=0 \in D$,
 
 \item предел $F(x_n)$ существует и равен $y^*=1 \in Y$
 
\end{enumerate}
верно равенство $F(x^*)=y^*$.

\end{example}

В определении замкнутости отображения исключаются сходящиеся последовательности $\lbrace x_n \rbrace$, для которых предел $F(x_n)$ не существует.

В этой главе мы будем рассматривать отображения, областями задания и областями значений которых являются линейные множества.

\begin{definition} Отображение $F$ называется \textbf{аддитивным}, если $F(x_1+x_2)=F(x_1)+F(x_2)$.
\end{definition}

\begin{definition} Отображение $F$ называется \textbf{однородным}, если $F(\lambda x)=\lambda F(x)$. Для комплексных линейных множеств $X$ и $Y$ выполнено: $F(ix)=iF(x)$.
\end{definition}

\begin{definition} Отображения аддитивные и однородные будем называть \textbf{операторами}. Для оператора $A$ верно:

\begin{center}
	$A(\lambda_1x_1+\lambda_2x_2)=\lambda_1Ax_1+\lambda_2Ax_2$
\end{center}

\end{definition}

В определении оператора требование непрерывности можно заменить непрерывностью на элементе $\mathbb{0} \in D(A)$: для любого элемента $x_0\in D(A)$ рассмотрим $x \to x_0$ и 
\begin{center}
$z=(x-x_0)\to \mathbb{0}$.
\end{center}
Тогда 
\begin{center}
$A(x-x_0)\to \mathbb{0}$ и $Ax \to Ax_0$.
\end{center}
 
Важнейшим классом операторов являются ограниченные операторы.

\begin{definition} Оператор $A$ называется ограниченным, если любое ограниченное множество он отображает в множество, ограниченное в $Y$.
\end{definition}

В дальнейшем мы будем рассматривать нормированные пространства $X$ и $Y$.

Для ограниченного оператора $A$  обозначим величину 
\begin{center}
$\underset{\Vert x\Vert=1}{\sup}\Vert Ax \Vert =\underset{S_1}{\sup}\Vert Ax \Vert =C_0<+\infty$
\end{center}

Тогда
\begin{center}
$\underset{S_R}{\sup}\Vert Ax \Vert=R\cdot C_0$ 
и величина 
$\underset{S_R}{\sup}\Vert Ax \Vert\to\infty $ 
при $R\to\infty$.
\end{center}
Линейная зависимость
\begin{center}
$\underset{S_R}{\sup}\Vert Ax \Vert$ 
\end{center}
от величины $R$ для ограниченных операторов даёт более практичное определение ограниченного оператора.

\begin{definition} Оператор $A$ называется линейным оператором из $X$ в $Y$, если величина 
\begin{center}
$ C_0=\underset{S_1}{\sup}\Vert Ax \Vert<+\infty $.
\end{center}
\end{definition}

Эта величина называется \textbf{нормой} линейного оператора, она обозначается
\begin{center}
$\lVert A \rVert,$ \quad
( $\lVert A \rVert_{X \to Y}$).
\end{center}

Величина $\underset{\lVert x \rVert \leqslant 1}{\sup}\lVert Ax \rVert = \lVert A \rVert = \underset{\lVert x \rVert = 1}{\sup}\lVert Ax \rVert$.

Действительно,
\begin{center}
$C_0 \leqslant \underset{\lVert x \rVert \leqslant  1}{\sup}\lVert Ax \rVert$.
\end{center}

С другой стороны 
\begin{center}
$\Vert Ax \Vert=
\Vert x \Vert \cdot \Vert A(\frac{x}{\Vert x \Vert}) \Vert \leqslant C_0$
\end{center} 
при $\Vert x \Vert \leqslant 1$; 
$\underset{\Vert x\Vert \leqslant 1}{\sup}\Vert Ax \Vert \leqslant C_0$.

Тогда 
\begin{center}
$ C_0 \geqslant\underset{\Vert x\Vert \leqslant 1}{\sup}\Vert Ax \Vert $, и мы получаем 
$\underset{\Vert x\Vert \leqslant 1}{\sup}\Vert Ax \Vert =\Vert A \Vert$.
\end{center}

Для линейных операторов верна оценка: $\Vert Ax \Vert _{Y} =\Vert A \Vert \cdot \Vert x \Vert_{X} $.

\textbf{Замечание.} Если получена оценка  $\Vert Ax \Vert \leqslant {C}  \Vert x \Vert$, то $\Vert A \Vert \leqslant {C} $.

\begin{example}
Интегральный оператор $K$ из $X=L_p(a,b)$ в $Y=L_q(a,b)$, $
\frac{1}{p}+\frac{1}{q}=1$,
\begin{center}
$y=K\cdot x$ , $ y(t)=\displaystyle\int\limits_a^b K(t,\tau)\cdot x(\tau)d\tau$.
\end{center}

Относительно ядра $K(t,\tau)$ будем предполагать, что 
\begin{center}
${(\displaystyle\int\limits_a^b \displaystyle\int\limits_a^b {\vert K(t,\tau)\vert} ^q dtd\tau)^\frac{1}{q}} <+\infty$.
\end{center}

Согласно неравенству Гёльдера интеграл 
\begin{center}
$\vert \displaystyle\int\limits_a^b K(t,\tau)\cdot x(\tau)d\tau \vert \leqslant (\displaystyle\int\limits_a^b {\vert K(t,\tau)\vert} ^q d\tau)^\frac{1}{q} \cdot {(\displaystyle\int\limits_a^b {\vert x(\tau)\vert} ^p d\tau)^\frac{1}{p}=}$

$= \Vert x \Vert _{L_p(a,b)}\cdot (\displaystyle\int\limits_a^b {\vert K(t,\tau)\vert} ^q d\tau)^\frac{1}{q}$.

${\Vert y \Vert}^q _{L_q(a,b)}=\displaystyle\int\limits_a^b {\vert \displaystyle\int\limits_a^b K(t,\tau)\cdot x(\tau)d\tau \vert^{q}} dt\leq {\Vert x \Vert}^q _{L_p(a,b)} \cdot \displaystyle\int\limits_a^b dt \displaystyle\int\limits_a^b {\vert K(t,\tau)\vert} ^q d\tau$.

${\Vert y \Vert}_{L_q(a,b)}=
{\Vert Kx\Vert}_{L_q(a,b)} \leqslant {(\displaystyle\int\limits_a^b \displaystyle\int\limits_a^b {\vert K(t,\tau)\vert} ^q d\tau dt)^\frac{1}{q}}\cdot \Vert x \Vert _{L_p(a,b)} $ 
\end{center}
и $\Vert K \Vert \leqslant {(\displaystyle\int\limits_a^b \displaystyle\int\limits_a^b {\vert K(t,\tau)\vert} ^q d\tau dt)^\frac{1}{q}}$.

Можно показать, что $\Vert K \Vert = 
{(\displaystyle\int\limits_a^b \displaystyle\int\limits_a^b {\vert K(t,\tau)\vert} ^q d\tau dt)^\frac{1}{q}}.$
\end{example}

\section{Пространство линейных операторов}
Рассмотрим множество всех линейных операторов их нормированного пространства $X$ в нормированное пространство $Y$. Множество таких операторов обозначим $\mathcal{L}(X,Y)$. Введем в этом множестве операции сложения и умножения на число:

$$A+B: (A+B)x = Ax+Bx, \lambda A: (\lambda A) = \lambda Ax$$

Ясно, что выполнены все аксиомы линейного множества, множество $\mathcal{L}(X,Y)$ - линейное множество. Введем метрику в этом множестве $\rho(A, B) = \Vert A-B\Vert$ и получим нормированное пространство $\mathcal{L}(X,Y)$. Проверим, например, выполнение неравенства треугольника. Согласно определению $(A+B)x = Ax+Bx$ и $\Vert(A+B)x\Vert\le(\Vert A\Vert+\Vert B\Vert)\Vert x\Vert$. Следовательно $\Vert A+B\Vert\le\Vert A\Vert+\Vert B\Vert$.

Определим теперь понятие сильной сходимости последовательности линейных операторов.

\begin{definition}
Последовательность $\{A_n\}\in \mathcal{L}(X,Y)$ \textbf{сильно сходится} к линейному оператору $A\in \mathcal{L}(X,Y)$, если $||A_n-A||\to 0$ при $n\to {\infty}$.
\end{definition}

\begin{theorem}
Если $Y$ --- банахово пространство, то пространство $\mathcal{L}(X,Y)$ также банахово.
\end{theorem}

\begin{proof}
Пусть $\{A_n\}$ - фундаментальная последовательность линейных операторов: $\Vert A_{n+m}-A\Vert\le \varepsilon$ при достаточно больших $n$. Взяв произвольный элемент $x\in X$, построим последовательность $y_n\in Y$, $y_n=A_nx$. Эта последовательность сходится в себе $\Vert y_{n+m}-y_n\Vert\le \Vert A_{n+m}-A_n\Vert \Vert x\Vert\le \varepsilon \Vert x\Vert$, и в силу полноты пространства $Y$ существует предельный элемент $y\le Y$, такой что $y=\lim_{} A_nx$ при $n \to \infty$. Следовательно определен оператор $A: y=Ax$.

Покажем, что этот оператор ограничен. Так как $\lvert \Vert A_{n+m} \Vert-\Vert A_n\Vert \rvert \leqslant \Vert A_{n+m}-A_n\Vert \leqslant \varepsilon$ при достаточно больших $n$, то числовая последовательность норм $\Vert A_n\Vert$ имеет предел $\lim_{n\to \infty} \Vert A_n\Vert=C$. Тогда $\Vert y_n\Vert\le\Vert A_n\Vert \Vert x\Vert$ и переходя к пределу, получим $\Vert y\Vert\le C\Vert x\Vert$, и норма оператора $A$ ограничена, оператор $A$ - линейный оператор, $\Vert A\Vert\le C$, $A\in \mathcal{L}(X,Y)$.
\end{proof}

Определим теперь понятие поточечной сходимости последовательности линейных операторов.

\begin{definition}
Последовательность линейных операторов $\{A_n\}\in \mathcal{L}(X,Y)$ сходится к оператору $A\in \mathcal{L}(X,Y)$ поточечно, если для любого $x\in X$ последовательность элементов $y_n=A_nx\to Ax$ при $n\to \infty$
\end{definition}

Из сильной сходимости $A_n\to A$ следует и поточечная сходимость последовательности $\{A_n\}$.

\begin{center}
$\Vert A_nx-Ax\Vert_Y = \Vert(A_n-A)x\Vert\le \Vert A_n-A\Vert \Vert x\Vert\to 0$ при $n\to \infty$
\end{center}

Обратное утверждение неверно.

\begin{example}
Рассмотрим банахово пространство $l_1$ числовых последовательностей $x=(x_1,x_2,\ldots,x_n,\ldots)$ таких что числовой ряд $\displaystyle\sum_{k=0}^\infty \lvert x_k\rvert $ сходится и норма $\Vert x\Vert=\displaystyle\sum_{k=1}^\infty \lvert x_k\rvert $

Зададим последовательность аддитивных операторов $A_n$ из $l_1$ в $l_1$:
$A_nx=(x_1,x_2,x_3,\ldots,x_n)$

Операторы $A_n$ линейные:
$\Vert A_n\Vert=\sup \limits_{S_1} \Vert A_nx\Vert=\displaystyle\sum_{k=1}^\infty \lvert x_k\rvert =1$, $\Vert A_n\Vert=1$, $A_n\in \mathcal{L}(l_1,l_1)$.

Рассмотрим последовательность линейных операторов $E-A_n\in \mathcal{L}(l_1,l_1)$. Ясно, что $\Vert(E-A_n)x\Vert=\Vert(0,0,\ldots,0,x_{n+1},x_{n+2},\ldots)\Vert=\displaystyle\sum_{k=n+1}^\infty \lvert x_k\rvert  \to 0$ при $n\to \infty$, так как числовой ряд $\displaystyle\sum_{k=1}^\infty \lvert x_k\rvert $ сходится.

Таким образом $A_nx\to x$ для любых $x\in l_1$ при $n\to \infty$. Последовательность операторов $A_n$ сходится поточечно к тождественному оператору.

На сфере $S_1$ в $l_1$ рассмотрим множество элементов $S_1^0$, таких что $x=(0,0,\ldots,0,x_{n+1},x_{n+2},\ldots)$, $\displaystyle\sum_{k=n+1}^\infty \lvert x_k\rvert=1$.
$\Vert E-A_n\Vert=\sup \limits_{S_1} \Vert(E-A_n)x\Vert\ge \sup \limits_{S_1^0} \Vert(E-A_n)x\Vert=\displaystyle\sum_{k=n+1}^\infty \lvert x_k\rvert=1$ и $\Vert E-A_n\Vert\ge 1$ при всех $n$. Последовательность линейных операторов $A_n$ не сходится сильно к тождественному оператору.

\end{example}

\section{Теорема Банаха-Штейнгауза}
Рассмотрим последовательность линейных операторов $A_n \subset \mathcal{L}(X,Y)$. Пусть для каждого $x \subset X\ \exists \lim\limits_{n\to \infty}A_n x = y(x) \subset Y$. Тем самым определен аддитивный и непрерывный оператор $A$, $Ax = y(x)$. Пример предыдущего параграфа показывает, что этот оператор может и не являться сильным пределом последовательности операторов $A_n$: хотя $A_n x \to E x$, но $\lim A_n \ne E$.

Вопрос: при каких условия существует \textbf{линейный} оператор $A$ и можно ли оценить его норму?

Мы начнем с простой леммы.

\begin{lemma}[I]
Пусть оператор $A_n \subset \mathcal{L}(X,Y)$ и известно, что для всех элементов некоторого шара $S_r(x_0)$ нормы $\lVert A_n x\rVert_Y$ ограничены: $\lVert A_n x\rVert \leqslant c$. Тогда существует постоянная $M$, такая что норма оператора $A_n$ ограничена числом $M$: $\lVert A_n \rVert \leqslant M$.
\end{lemma}

\begin{proof}
Возьмем любой элемент $x \subset X$ и образуем элемент $x_0+\frac{x}{\lVert x\rVert}r \subset S_r(x_0)$. Тогда:

$$\lVert A_n(x_0+\frac{r}{\lVert x \rVert}x) \rVert \leqslant c \mbox{, т.е.}$$

\begin{align*}
c &\geqslant \lVert A_n(x_0+\frac{r}{\lVert x \rVert}x) \rVert,\\
c &\geqslant \lVert \frac{r}{\lVert x \rVert}A_n x + A_n x_0 \rVert \geqslant \lVert A_n x \rVert \frac{r}{\lVert x\rVert} - \lVert A_n x_0 \rVert \geqslant \frac{r}{\lVert x \rVert}\lVert A_n x_0 \rVert - c,\\
2c &\geqslant \frac{r}{\lVert x \rVert}\lVert A_n x\rVert,\\
\frac{2c}{r}\lVert x \rVert &\geqslant \lVert A_n x\rVert,\\
\lVert A_n x \rVert &\leqslant \frac{2c}{r}\lVert x \rVert.
\end{align*}

Достаточно положить $M=\frac{2c}{r}$ : $\lVert A_n x\rVert \leqslant M\lVert x\rVert$ для всех $x \subset X$, следовательно норма оператора $\lVert A\rVert \leqslant M$.
\end{proof}

Следующая лемма верна для полных нормированных пространств $X$, т.е. для банаховых пространств.

\begin{lemma}[II]
Пусть $X$ --- пространство Банаха. Пусть $\lbrace A_n\rbrace$ последовательность операторов множества $\mathcal{L}(X,Y)$. Тогда если нормы $\lVert A_n x\rVert$ элементов $A_n x$ ограничены для каждого $x \subset X$, то нормы операторов $A_n \quad \lVert A_n\rVert$ ограничены в совокупности.
\end{lemma}

\begin{proof}
(От противного)

Предположим, что в $X$ существует замкнутый шар $\bar{S_0}$, для \textbf{всех} элементов которого $\lVert A_n x\rVert < c$ при всех $n$, но последовательность норм $\lVert A_n\rVert$ неограниченна. Это предположение неверно, так как согласно Лемме I норма каждого оператора $\lVert A_n\rVert \leqslant M$.

Остаётся предположить, что существует шар $\bar{S_0}$, в котором значения $\lVert A_n x\rVert$ неограниченны, т.е. найдется номер $n_1$ и найдется элемент $x_1 \subset \bar{S_0}$, такие что $\lVert A_{n1} (x_1)\rVert > 1$. Так как оператор $A_{n1}$ (как и все операторы $A_n$) непрерывен, то существует шар $\bar{S_1} \subset \bar{S_0}$, в котором $\lVert A_{n1} (x)\rVert > 1$.

Далее: значения $\lVert A_n x\rVert$ в шаре $\bar{S_1}$ не могут быть ограничены при всех $n$. Тогда должен существовать номер $n_2$ и элемент $x_2 \subset \bar{S_2}$, такие что $\lVert A_{n2} (x_2)\rVert > 2$, (a по непрерывности $A_{n2}$) и шар $S_2$, в котором $\lVert A_{n2} (x)\rVert > 2$ для всех $x \subset \bar{S_2}$.

Продолжая этот процесс мы получим последовательность вложенных шаров $\bar{S_0} \supset \bar{S_1} \supset \bar{S_2} \ldots \supset \bar{S_n} \supset \ldots$ и последовательность элементов $x_0, x_1, x_2, \ldots, x_n$, таких что $\lVert A_{nk} x_k\rVert > k$. Ясно, что радиусы $r_k$ шаров $\bar{S_k}$ можно выбирать так, что $r_k \to 0$.

Так как пространство $X$ полное, то по теореме о вложенных шарах существует элемент $x^k \subset X$ принадлежащий всем шарам $S_k$. Тогда $\lVert A_n x_n\rVert \to \infty$ при $k \to \infty$, что противоречит условиям леммы.
\end{proof}

\begin{remark}
Если $\lbrace \lVert A_n x\rVert \rbrace$ не ограничена, то существует элемент $x^* \subset X$, на котором $\sup\limits\lVert A_n x^*\rVert = \infty$ --- принцип фиксации особенности в полном банаховом пространстве $X$.
\end{remark}

Теперь можно указать условия, при которых из сходимости последовательности элементов $\lbrace A_n x\rbrace$ при любом $x \subset X$ следует поточечная сходимость последовательности операторов $A_n$ к \textbf{линейному} оператору $A$: $\lim\limits_{n \to \infty}A_n x = y(x) \subset Y$, $Ax=y(x)$.

\begin{theorem}[Банаха-Штейнгауза]
Пусть $\lbrace A_n\rbrace \subset \mathcal{L}(X,Y)$, где $X$ и $Y$ --- пространства Банаха. Для того, чтобы последовательность операторов $\lbrace A_n\rbrace$ сходилась поточечно к линейному оператору на всем пространстве $X$, необходимо и достаточно, чтобы были выполнены два условия:

\begin{enumerate}
\item Нормы $\lVert A_n\rVert$ ограничены в совокупности: $\lVert A_n\rVert \leqslant M$.
\item Последовательность элементов $A_n x$ сходится в себе на множестве $D$ плотном в $X$.
\end{enumerate}
\end{theorem}

\begin{proof}
\underbar{Необходимость} Пусть $A_n x \to A x$. Тогда $\lVert A_n x\rVert \to \lVert A x\rVert \leqslant \lVert A\rVert \lVert x\rVert$, и последовательность норм элементов $\lVert A_n x\rVert$ ограничена при каждом $x \subset X$. По Лемме II нормы $\lVert A_n\rVert$ ограничены в совокупности --- условие 1 выполнено.

Так как последовательность $A_n x$ сходится на всем пространстве $X$ (и сходится в себе на всем $X$), то она сходится в себе на любом множестве пространства $X$ --- условие 2 выполнено.

\underbar{Достаточность} Покажем, что последовательность элементов $A_n x$ сходится в себе на всем пространстве $X$.

По любому заданному $\varepsilon > 0$ для любого элемента $x \subset X$ найдется элемент $x' \subset D$ такой, что $\lVert x - x'\rVert < \varepsilon$. Оценим $\lVert A_n x - A_m x\rVert$ при достаточно больших значениях $m$ и $n$:

\begin{center}
$\lVert A_n x - A_m x\rVert = \lVert A_n x' - A_m x' + A_n x - A_n x' + A_m x' - A_m x\rVert$
\end{center}

Значения $\lVert A_n x' - A_m x'\rVert < \varepsilon$ при достаточно больших $m$ и $n$.

Значения $\lVert A_n x - A_n x'\rVert < M\varepsilon$, $\lVert A_m x' - A_m x\rVert < M\varepsilon$.

Итак, $\lVert A_n x - A_m x\rVert < (2M +1)\varepsilon$ при достаточно больших $m$ и $n$, последовательность $A_n x$ сходится в себе на всем $X$, а так как пространство $Y$ полное, то существует $\lim\limits_{n \to \infty} A_n x = y(x) \subset Y$ и тем самым определен оператор: $A x = y(x)$. Для оценки нормы этого оператора из неравенства $\lVert A_n x\rVert < M\lVert x\rVert$ получим $\lVert A\rVert \geqslant \overbar{\lim\limits_{n \to \infty}}\lVert A_n\rVert$ ($\overbar{\lim\limits_{n \to \infty}}a_n = a$, если для любого $\varepsilon > 0$ в интервале $(a-\varepsilon, a+\varepsilon)$ содержится бесконечное число элементов последовательности $\lbrace a_n\rbrace$, а справа от этого интервала существует не более чем конечное число членов последовательности $\lbrace a_n\rbrace$).

Напомним, что последовательность операторов $A_n$ может и не иметь сильного предела.
\end{proof}

\begin{remark}
Формулировка теоремы не предполагает знания оператора $A$. Если же оператор $A$ предъявлен, то вместо условия 2 можно проверить сходимость $A_n x\to A x$ на линейном множестве $D$ плотном в $X$. В этом случае предположение, что $Y$ банахово --- излишне.
\end{remark}

\section{Теорема Банаха}
Основной целью нашего курса является изучение условий разрешимости уравнения $Ax=y$, т. е. условие существования $\underbar {линейного}$ оператора $B\in \mathcal{L}(Y, X)$: $By=A^{-1}x$. Если такой оператор существует, то задача решения уравнения $Ax=y$ поставлена корректно по Адамару:
\begin{enumerate}
	\item решение существует для любого $y \in Y$,
	\item решение единственно,
	\item вариации $\Delta x$ решения непрерывно зависят от вариаций $\Delta y$ элемента $y$: $\lVert \Delta x\rVert_X \leqslant \lVert B\rVert_{Y\to X}\lVert \Delta y\rVert_X$.
\end{enumerate}
Оператор $A$ предполагается линейным, а оператор $B$ определен на всем банаховом пространстве $Y$. В этом случаем операторы $A$ и $B$ замкнуты. Действительно:
\begin{enumerate}

	\item Из условий $\{x_n \to x^{*};Ax_n \to y^{*}\}$ следует, что $Ax_n \to Ax^{*}$ и $y^{*}=Ax^{*}$, т.е. оператор $A$ замкнут.

	\item Пусть выполнены условия: $\{y_n \to y^{*}; By_n \to x^{*}\}$. Обозначим $x_n=A^{-1}y_n$ (оператор $A^{-1}$ определен на всём пространстве $Y$). Тогда $x_n \to x^{*}$ и $y_n=Ax_n \to Ax^{*}$. Так как оператор $A$ замкнут, то из условий $\{x_n \to x^{*};Ax_n \to y^{*}\}$ следует, что $y^{*}=Ax^{*}$, т.е. $x^{*}=By^{*}$. Тогда оператор $B$ замкнут.
	
\end{enumerate}
	
\begin{lemma}
Пусть оператор $B$ задан на всем $\underbar {банаховом}$ пространстве $Y$. Тогда в $Y$ существует плотное множество элементов $M$, таких, что $\lVert By\rVert_X \leqslant C_0\lVert y\rVert_Y $, где значение постоянной $C_0$ не зависит от выбора элементов $y \subset M$.
\end{lemma}

\begin{proof}
	
Построим множество $M$.

\begin{enumerate}
	\item Для любого выбранного элемента $y \subset Y$ можно найти такое натуральное число $k$, что $\lVert By\rVert \leqslant k\lVert y\rVert$.
	
	Множество таких элементов $y$ обозначим $Y_k$. Множество $Y_k$ <<однородно>>: если $y \subset Y_k$, то элементы $\lambda y \subset Y_k$ для всех чисел $\lambda$. Ясно, что при $m \leqslant k$ $Y_m \subset Y_k$ и что

\begin{center}	

	$Y=\bigcup\limits_{n=1}^\infty Y_n$
	
\end{center}
	
	Так как $Y$ полное пространство, то среди множеств $Y_k$ найдется хотя бы одно множество, которое не является нигде не плотным в пространстве $Y$ (см. теорему Бэра, глава \uppercase\expandafter{\romannumeral 1}). Обозначим его $Y_n$.
	
	\item В пространстве $Y$ существует хотя бы один шар (например $S$), в котором любой шар (например $\overbar{S}_{r_0}(u_0)$) содержит элементы множества $Y_n$.
	
	Таким образом, множество $Y_n$ плотно в $\overbar{S}_{r_0}(u_0)\subset S$. Можно считать, что элемент $u_0 \subset Y_n$. Зафиксируем элемент $u_0$ и число $r_0$.
	
	\item Пусть $y$ любой элемент $Y$, норма которого равна $r_0:\lVert y\rVert=r_0$. Построим элемент $z_0=u_0+y$, принадлежащий границе шара $S_{r_0}(u_0)$. Далее построим последовательность элементов $u_j \subset S_{r_0}(u_0)$ таких, что

\begin{center}	

	$u_j \to z_0$ и $u_j \subset Y_n$.
	
\end{center}

Обозначим $y_j=u_j-u_0$. Ясно, что $y_j \to y$, $y_j \subset S_{r_0}(u_j)$ (последовательность элементов $y_j$ не обязательно принадлежит множеству $Y_n$!)
	
\end{enumerate}


\begin{enumerate}
\item Свойство элементов $y_j$.

$$\lVert By_j\rVert = \lVert B(u_j - u_0)\rVert =$$ (аддитивность $B$) 
$$ = \lVert Bu_j - Bu_0\rVert \leqslant \lVert Bu_j\rVert + \lVert Bu_0\rVert(u_0, u_j \in Y_n) \leqslant n(\lVert u_j\rVert + \lVert u_0\rVert)\cdot 1.$$
Оценим $\lVert B_j\rVert$ через норму элементов $\lVert y_j\rVert$. Так как $y_j \to y$, то $\lVert y_j\rVert \to \lVert y\rVert=r_0$, и при достаточно больших значениях $j$:

\begin{center}
$\lVert y_j\rVert\frac{1}{r_0}>\frac12$ и $1<\frac{2}{r_0}\lVert y_j\rVert$.
\end{center}

Для таких значений $j$: $\lVert By_j\rVert\le\frac{2}{r_0}n(\lVert u_j\rVert + \lVert u_0\rVert)$ и так как $\lVert u_j\rVert = \lVert u_0 + u_j - u_0\rVert \leqslant \lVert u_0\rVert + \lVert u_j - u_0\rVert \leqslant \lVert u_0\rVert + r_0$, то для достаточно больших значений $j$:

\begin{center}
$\lVert By_j\rVert_X \leqslant \frac{2}{r_0}(r_0 + \lVert u_0\rVert)\cdot n\cdot \lVert y_j\rVert_Y$.
\end{center}

Величину $\frac{2}{r_0}(r_0 + \lVert u_0\rVert)\cdot n$ оценим натуральным числом $N$:

\begin{center}
$\lVert By_j\rVert \leqslant N\lVert y_j\rVert$ и $y_j \in Y_N$.
\end{center}

Так как $y_j \to y$, то множество $Y_N$ плотно в множестве элементов $y$ с нормой $r_0$. Значение $N$, согласно п.п. 1.2, зависит только от фиксированных значений $r$, $n_0$ и $u_0 \in Y$. Так как множество $Y_N$ ``однородно'', то и для любого $y \in Y$ существует последовательность элементов $y_j \in Y_N$, такая что $y_j \to y$ при $j \to \infty$ и $\lVert By_j\rVert \leqslant N\lvert y_j\rVert$. Обозначив $M = Y_N$ и $C_0 = N$, завершим доказательство леммы.
\qedhere
\end{enumerate}
\end{proof}

\begin{theorem}[Банаха]
Замкнутый оператор $B$, действующий из банахова пространства $Y$ в банахово пространство $X$ и определённый на всём пространстве $Y$, линеен.
\end{theorem}

\begin{proof}
Согласно лемме в пространстве $Y$ существует всюду плотное множество $M$, для элементов которого
\begin{center}
	$\Vert By \Vert \leqslant C_0 \Vert y \Vert $
\end{center}
\begin{enumerate}

	\item Пусть $y_0$ любой элемент пространства $Y$.
	
	Построим шар радиуса $\frac{1}{4}\Vert y_0 \Vert$ с 		центром $\frac{3}{4}y_0$. В этом шаре найдём элемент
	$y_1 \in M$:
	\begin{center}
	$\Vert y_1 - \frac{3}{4}y_0\Vert \leqslant \frac{1}{4}    		\Vert y_0 \Vert $,
	\end{center}
	\begin{center}
	$\Vert y_1 \Vert =
	\Vert y_1 - \frac{3}{4}y_0 +
	\frac{3}{4}y_0 \Vert \leqslant \frac{1}{4} 			\Vert y_0 \Vert +
	 \frac{3}{4}\Vert y_0\Vert = 				\Vert y_0 \Vert$
	\end{center}
	
	\item Рассмотрим элемент $y_1 - y_0$. Для него
	\begin{center}
	$\Vert y_1-y_0 \Vert =\Vert y - \frac{3}{4}y_0 +
	\frac{1}{4}y_0 \Vert \leqslant
	\frac{1}{4} \Vert y_0 \Vert +
	\frac{1}{4} \Vert y_0 \Vert =
	\frac{1}{2} \Vert y_0 \Vert$
	\end{center}
	Построим шар радиуса $\frac{1}{4} \Vert 		y_1 - y_0 \Vert$ с центром $\frac{3}{4} (y_1 - y_0)$. В этом шаре найдём элемент $y_2 \in M$:
	\begin{center}
	$\Vert y_2 + y_1 - y_0 \Vert =
	\Vert y_2 - \frac{3}{4}(y_1 - y_0) -
	\frac{1}{4}(y_1 - y_0) \Vert \leqslant
	\frac{1}{4} \Vert y_1 - y_0 \Vert +
	\frac{1}{4} \Vert y_1 - y_0 \Vert = $
	\end{center}
	\begin{center}
	$ = \frac{1}{2} \Vert y_1 - y_0 \Vert 			\leqslant
	\frac{1}{2^2} \Vert y_0 \Vert$ ,
	\end{center}
	\begin{center}
	$\Vert y_2 \Vert =
	\Vert y_2 - \frac{3}{4}(y_1 - y_0) +
	\frac{3}{4}(y_1 - y_0) \Vert \leqslant
	\frac{1}{4} \Vert y_1 - y_0 \Vert +
	\frac{3}{4} \Vert y_1 - y_0 \Vert =
	 \Vert y_1 - y_0 \Vert \leqslant$
	 \end{center}
	\begin{center}
	 $ \leqslant \frac{1}{2} \Vert y_0 \Vert$
	\end{center}
	
	\item Рассмотрим элемент $y_2+y_1- y_0$ . Для него 	\begin{center}	
	$\Vert y_2+y_1- y_0 \Vert \leqslant
	\frac{1}{2^2} \Vert y_0 \Vert$ .
	\end{center}
	Построим шар радиуса
	$\frac{1}{4} \Vert y_2 + y_1 - y_0 \Vert$      	с центром $\frac{3}{4}(y_2 + y_1 - y_0)$.
	В этом шаре найдём элемент 	$y_3 \in M$:
	\begin{center}
	$\Vert y_3 + y_2 + y_1 - y_0 \Vert =
	\Vert y_3 - \frac{3}{4}(y_0 - y_1 - y_2) -
	\frac{1}{4}(y_0 - y_1 - y_2) \Vert \leqslant $
	\end{center}
	\begin{center}
	$\leqslant \frac{1}{4} \Vert y_0 - y_1 - y_2 \Vert +
	\frac{1}{4} \Vert y_0 - y_1 - y_2 \Vert 			\leqslant $
	\end{center}
	\begin{center}
	$ \leqslant\frac{1}{2} \cdot
	\frac{1}{2^2} \Vert y_0 \Vert =
	\frac{1}{2^3} \Vert y_0 \Vert$ ,
	\end{center}
	\begin{center}
	$\Vert y_3 \Vert =
	\Vert y_3 - \frac{3}{4}(y_0 - y_1 - y_2) +
	\frac{3}{4}(y_0 - y_1 - y_2) \Vert \leqslant $
	\end{center}
	\begin{center}
	$\leqslant \frac{1}{4} \Vert y_0 - y_1 - y_2 				\Vert +
	\frac{3}{4} \Vert y_0 - y_1 - y_2 \Vert =
	 \Vert y_0 - y_1 - y_2 \Vert \leqslant
	 \frac{1}{2^2} \Vert y_0 \Vert$
	\end{center}

Продолжая этот процесс, получим элементы $y_n, y_{n-1}, y_{n-2}, \ldots, y_3, y_2, y_1 \in M$ такие что 

\begin{center}
$\lVert y_n \rVert \leqslant \frac{1}{2^{n-1}} \lVert y_0 \rVert$

и

$\lVert y_n + y_{n-1}+y_{n-2}+\ldots+y_2+y_1+y_0 \rVert \leqslant \frac{1}{2^n} \lVert y_0 \rVert.$
\end{center}

Обозначим $s_n= \displaystyle\sum\limits_{k=1}^n y_k$. Тогда $s_n \to y_0$ при $n \to \infty$. Последовательность элементов $B s_n$ сходится в себе:

\begin{center}
$ \lVert B s_{n+m}-B s_n \rVert _X = \lVert B(s_{n+m}-s_n) \rVert= \lVert B(y_{n+1}+y_{n+2}+\ldots+y_{n+m}) \rVert \leqslant $
\end{center}

\begin{center}
$\leqslant C_0 \lVert y_0 \rVert (\frac{1}{2^n} + \frac{1}{2^{n+1}}+\ldots+\frac{1}{2^{n+m-1}})=C_0 \frac{1}{2^n} (1 + \frac{1}{2}+\ldots+\frac{1}{2^{m-1}}) \to 0$ при $n \to \infty$.
\end{center}

Так как пространство $X$ полное, то существует $\lim \limits_{n \to \infty} s_n=x^* \in X$.

Переходя в оценке $ \lVert B s_{n+m}-B s_n \rVert$ к пределу при $m \to \infty$, получаем

\begin{center}
$ \lVert x^*-B s_n \rVert \leqslant C_0 \lVert y_0\rVert (\frac{1}{2^n} + \frac{1}{2^{n+1}}+\ldots)$.
\end{center}

Так как $s_n \to y_0, B s_n \to x^*$, то в силу замкнутости оператора $B$: 

\begin{center}
$B y_0=x^* $.
\end{center}

Оценим $\lVert B y_0 \rVert _X$:

\begin{center}
$\lVert B y_0 \rVert \leqslant \lVert x^* - B s_n \rVert + \lVert B s_n \rVert \leqslant$ 
\end{center}

\begin{center}
$\leqslant C_0 \lVert y_0\rVert \Big[ \frac{1}{2^n} + \frac{1}{2^{n+1}}+\ldots \Big] + C_0 \lVert y_0\rVert \Big[ 1+\frac{1}{2} +\frac{1}{2^2}+ \ldots+\frac{1}{2^{n-1}} \Big]= 2 C_0 \lVert y_0\rVert$
\end{center}

$\lVert B y_0 \rVert \leqslant 2 C_0 \lVert y_0\rVert$, т.е. оператор $B$ линеен, $ \lVert B y_0 \rVert _{Y \to X} \leqslant 2 C_0$.
\qedhere
\end{enumerate}
\end{proof}

\begin{example}
В качестве примера рассмотрим обратную задачу теплопроводности.

\begin{enumerate}

\item Прямая задача теплопроводности.

Рассматривается бесконечная пластина толщиной $\pi$: $0 \leqslant \xi \leqslant\pi$, $-\infty < y < +\infty$. Распределение температуры $x(\xi, y, t)$ в точках этой пластины зависит только от координаты $\xi$ и времени $t$: $x(\xi, y, t) = x(\xi, t)$.На граничных плоскостях при $\xi = 0$ и при $\xi = \pi$ температура равна нулю при всех $t > 0$:

\begin{equation}
\label{border_conditions}
x(0, t) = 0,\;  x(\pi, t) = 0
\end{equation}

Внутри пластины источников тепла нет. Функция $x(\xi, t)$ удовлетворяет уравнению теплопроводности:

\begin{equation}
\label{heat_equation}
\frac{\partial x}{\partial t} = \frac{\partial^2 x}{\partial \xi^2} \;  \mbox{при} \;   \xi \in (0, \pi),\;  t > 0
\end{equation}

В момент времени $t = 0$ распределение температуры известно: 

\begin{equation}
\label{initial_conditions}
x(\xi,0)=x(\xi)
\end{equation}

Прямая задача теплопроводности состоит в нахождении функции $x(\xi,t)$ при всех $t >0$, т.е. в решении уравнения (\ref{heat_equation}) при граничных условиях (\ref{border_conditions}) и при начальном условии (\ref{initial_conditions}).

\guillemotleft Обобщенное\guillemotright \;решение этой простейшей задачи при условии $x(\xi) \in L_2(0, \pi)$ получаем методом Фурье:
\begin{center}
$$x(\xi,t) = \displaystyle\sum\limits_{k=1}^n x_k e^{-k^2t} X_k(\xi),$$ где $X_k(\xi) = \sqrt{\frac{2}{\pi}} \sin (k\xi)$, а числа $x_k = (x,X_k) = \displaystyle\int\limits_0^\pi x(\xi) X_k(\xi) d\xi\mbox{.}$
\end{center}

Ясно, что $\|x(\cdot, t)\|_{L_2 (0, \pi)}^{2} \leqslant \displaystyle\sum\limits_{k=1}^{\infty}x_k^2=\|x\|^2$, т.е. функция $x(\xi,t)$ как функция переменной $\xi$ принадлежит пространству $L_2 (0, \pi)$.
\begin{center}
$$x(\xi, t) = \displaystyle\sum\limits_{k=1}^{\infty} e^{-k^2 t} X_k(\xi) \int\limits_{0}^{\pi} x(\eta) X_k(\eta) d\eta = \int\limits_{0}^{\pi} K(\xi, \eta, t) x(\eta) d \eta \mbox{.}$$
\end{center}

В частности $x(\xi,T) = \displaystyle\int\limits_0^\pi K(\xi, \eta, T) x(\eta)d\eta = Ax$. Оператор $A$ линейный интегральный оператор из пространства $L_2(0, \pi)$ в пространство $L_2(0, \pi)$. Норма этого оператора не превосходит $1$, оператор $A$ замкнут.

	\item Обратная задача теплопроводности.

В момент времени $t = T$ измеряется  распределение температуры:
\begin{center}
$x(\xi, T) = y(\xi)$
\end{center}

Требуется \guillemotleft восстановить\guillemotright \;неизвестное начальное распределение $x(\xi)$, т.е. требуется решить уравнение $Ax = y$, где $A \in \mathcal{L}(X,Y), \; X = L_{2}(0, \pi), \; Y = L_{2}(0, \pi)$ --- банаховы пространства.

Будет ли эта задача корректно поставлена по Адамару?

Довольно просто доказать, что если решение задачи существует, то это решение единственно. Для того, чтобы задача была поставлена корректно по Адамару, остается доказать (согласно теореме Банаха), что задача разрешима при любой функции $y \in Y$, т.е. доказать, что обратный оператор $B = A^{-1}$ заданы на всем пространстве $Y = L_{2}(0, \pi)$.

Это утверждение неверно.

Действительно, рассмотрим функцию
\begin{center}
$y(\xi)=\displaystyle\sum\limits_{k=1}^{\infty} \frac{1}{k} X_k(\xi) ,  \; y\in L_{2}(0, \pi), \; \|y\|^2 = \sum\limits_{k=1}^{\infty} \frac{1}{k^2} = \frac{\pi^2}{6} ,  \; \|y\| = \frac{1}{\sqrt{6}}\pi \mbox{.}$
\end{center}

Предположим, что решение $x(\xi)$ такой обратной задачи существует. Тогда коэффициенты Фурье $x_k$ этого решения должны быть равными

$$x_k=e^{k^2 T} \frac{1}{k} \; \mbox{ и } \; x(\xi) = \displaystyle\sum\limits_{k=1}^{\infty} x_k X_k(\xi)$$

Но ряд $\displaystyle\sum\limits_{k=1}^{\infty} x_k^2 = \sum\limits_{k=1}^{\infty} e^{2k^2 T} \frac{1}{k^2}$ при $T > 0$ расходится, следовательно функция $x$ не принадлежит пространству $L_2(0,\pi)$. Наше предположение неверно.

Ясно, что отсутствует и непрерывная зависимость решения от вариации правых частей. Действительно, пусть $x^{*}$ есть точное решение уравнения $Ax^*=y^*$. Рассмотрим вариацию правой части  $\Delta y$:
\begin{center}
$$\Delta y = \varepsilon \int\limits_{k=1}^{n} \frac{1}{k} X_k(\xi) , \; \; \| \Delta y\|^2 = \varepsilon ^2 \sum\limits_{k=1}^{n} \frac{1}{k^2} < \varepsilon^2 \frac{\pi ^2}{6} \; \; \mbox{и} \; \; \| \Delta y \| < \frac{\varepsilon}{\sqrt{6}}\pi.$$
\end{center}

Для соответствующей вариации решения $\Delta x$ получаем
\begin{center}
$$\| \Delta x \|_{L_2(0, \pi)}^2 = \varepsilon ^2 \sum\limits_{k=1}^n \frac{1}{k^2} e^{2 k^2 T},$$
\end{center}
и величина $\| \Delta x \|$ сколь угодно велика при $n \longrightarrow \infty.$

\end{enumerate}

\end{example}

\section{Вполне непрерывные операторы}
\begin{definition}Линейный оператор $A \subset \mathcal{L}(X, Y)$ называется \textbf{вполне непрерывным}, если любое ограниченное в $X$ множество он отображает в множество, компактное в $Y$.
\end{definition}
Напомню, что в компактном множестве в любой последовательности $\lbrace y_n \rbrace_{n=1}^{\infty}$ содержится фундаментальная последовательность. Если же пространство $Y$ полно, то согласно определению эта фундаментальная последовательность имеет предел в $Y$.

Множество всех вполне непрерывных операторов обозначим $\sigma(X,Y)$.
\begin{theorem}Множество $\sigma(X,Y)$ является подпространством пространства $\mathcal{L}(X, Y)$.
\end{theorem}
\begin{proof}Состроит в доказательстве двух пунктов (согласно определению подпространства).

\begin{enumerate}

	\item Если $A_1, A_2 \subset \sigma(X,Y)$, то их линейная комбинация $A=\lambda_1 A_1+\lambda_2 A_2\in \sigma(X,Y)$.

Рассмотрим множество $AS_1$, где $S_1$ --- единичная сфера в пространстве $X$. Покажем, что множество $AS_1$ компактно в $Y$. Возьмем любую последовательность элементов $x_n \subset S_1$, $\lVert x_n \rVert=1$. Обозначим элементы $Ax_n=y_n$, $y_n=\lambda_1 A_1 x_n+\lambda_2 A_2 x_n$.

Так как множество $A_1S_1$ компактно в $Y$, то из последовательности $\lbrace A_2x_{n_k} \rbrace$ можно выделить фундаментальную последовательность $\lbrace A_2x'_i \rbrace$. Ясно, что последовательность $\lbrace (\lambda_1 A_1+\lambda_2 A_2)x'_i \rbrace$ является фундаментальной последовательностью в $Y$.

	\item Покажем, что множество $\sigma(X,Y)$ замкнуто. Так как $\lVert A_n-A \rVert \to 0$ при $n\to \infty$, то для выбранного $\varepsilon > 0$ рассмотрим операторы $A_n$ такие что $\lVert A_n-A \rVert \leqslant  \varepsilon$ и при $x \subset S_1$ $\lVert A_n x-Ax\rVert \leqslant  \varepsilon$. Зафиксируем $n$. Рассмотрим множество элементов $A_n S_1$. Так как множество $A_n S_1$ компактно в $Y$, то в множестве $A_n S_1$ существует конечная $\varepsilon$-сеть $\lbrace y_k \rbrace$ $y_k=A_n x_k$, $x_k \subset S_1$. Тогда $\lVert y_k - Ax\rVert \leqslant \lVert y_k - A_n x\rVert + \lVert A_n x - Ax\rVert \leqslant 2\varepsilon$. Следовательно, элементы $y_k$ образуют $2\varepsilon$-сеть в множестве $Y$ и, согласно теореме Хаусдорфа, множество $AS_1$ компактно.
\qedhere
\end{enumerate}
\end{proof}

\begin{cexample}Тождественный оператор $E$ в сепарабельном гильбертовом пространстве не является вполне непрерывным оператором.

Действительно, пусть $\psi_1,\psi_2,\ldots,\psi_n,\ldots, \lVert\psi_n\rVert =1$ --- ортонормальный базис пространства. Множество $S_1$ ограничено, но множество $ES_1 (=S_1)$ не является компактным: из последовательности $\lbrace \psi_n \rbrace_{n=1}^{\infty}$ нельзя выбрать фундаментальную последовательность, так как $\lVert\psi_n - \psi_m \rVert ^2=(\psi_n - \psi_m, \psi_n - \psi_m)= (\psi_n,\psi_n)+ (\psi_m,\psi_m)=2$ при $n \neq m$
\end{cexample}

\begin{example}[1]
Интегральный оператор $\widetilde{K}$ из $L_2(a,b)$ в $L_2(a,b)$.

\begin{center}
$y=\widetilde{K}x$, $y(t)=\displaystyle\int\limits_a^b \widetilde{K}(t, \tau)x(\tau)d\tau$,
\end{center}

где ядро $\widetilde{K}(t, \tau)$ непрерывно в области $D=[a,b]\times[a,b]$, $|\widetilde{K}(t, \tau)|\leqslant M$.

В этом случае функции $y(t)$ непрерывны:

\begin{center}
$|y(t_2)-y(t_1)|^2\leq \displaystyle\int\limits_a^b |\widetilde{K}(t_2, \tau)-\widetilde{K}(t_1, \tau)|^2d\tau \cdot \displaystyle\int\limits_a^b |x(\tau)|^2 d\tau \leqslant \varepsilon^2 (b-a){\lVert x\rVert^2}_{L_2(a,b)}$
\end{center}

Величина $|y(t_2)-y(t_1)|\to 0$ при $|t_2-t_1|\to 0$ в силу непрерывности функции $\widetilde{K}(t, \tau)$ как функции двух переменных. Таким образом, множество функций $\widetilde{K} S_1$ равностепенно непрерывно.

Ясно, что функции множества $\widetilde{K} S_1$ ограничены в совокупности: $|y(t)|^2 \leqslant M^2 (b-a)$.

По теореме Арцела-Асколи множество $\widetilde{K} S_1$ компактно в $C[a,b]$: из любой последовательности элементов $y_n=\widetilde{K} x_n$ можно выделить фундаментальную последовательность в $C[a,b]$, которая является фундаментальной последовательностью и в пространстве $L_2 (a,b))$.
\end{example}

\begin{example}[2]
Интегральный оператор $K$ из $L_2(a,b)$ в $L_2(a,b)$.

$$y=Kx, \; y(t) = \displaystyle\int\limits_a^b K(t, \tau)x(\tau)d\tau,$$

где $K(t, \tau)\subset L_2(D)$ (интегральный оператор Гильберта-Шмидта).

По теореме Лебега для функции $K(t, \tau)$ существует последовательность непрерывных в $D$ функций $\widetilde{K}_n (t, \tau)$, таких что

$$\displaystyle\int\limits_a^b |K(t, \tau) - K_n(t, \tau)|^2d\tau dt \to 0 \mbox{ при } n\to \infty,$$

$$\mbox{т.е. } \lVert K-\widetilde{K}_n \rVert \to 0 \mbox{ при } n\to \infty.$$

Так как интегральные операторы $\widetilde{K}_n$ вполне непрерывны, то и интегральный оператор Гильберта-Шмидта вполне непрерывен.
\end{example}

\begin{theorem}
Пусть оператор $A \subset \sigma (H,H)$, где $H$ бесконечномерное сепарабельное пространство Гильберта. Задача решения уравнения $Ax=y$ поставлена некорректно по Адамару.
\end{theorem}

\begin{proof}
В этом случае легко доказать, что нарушено условие непрерывной зависимости решения при вариации первой части. Действительно, так как множество $AS_1$ компактно в $H$, то из последовательности $\lbrace A \psi_n\rbrace_{n=1}^{\infty}$ можно выбрать сходящуюся подпоследовательность элементов $\lbrace A \psi_{n_k}\rbrace_{n=1}^{\infty}$, а так как пространство Гильберта полное, то эта фундаментальная последовательность сходится к элементу $y_0 \subset H$: $y_0=\underset{k \to \infty}{\lim} A \psi_{n_k}$.

Уравнение $Ax = y_0$ не имеет решения: рассмотрим уравнения $Ax = A \psi_{n_k}$. Решения существуют и единственны: $x_k = \psi_{n_k}$. Предел в правой части $\displaystyle\lim_{k \to \infty} A \varphi_k = y_0 \in H$ существует. Но предел решения $\displaystyle\lim_{k \to \infty} x_k$ не существует.
\end{proof}

\chapter{Продолжение линейных операторов}

\section{Продолжение линейных операторов}
Пусть линейный оператор $A_0\in \mathcal{L}(X_0,Y)$, где $X_0$ есть подпространство пространства $X$. Можно ли продолжить оператор $A_0$ на все пространство $X$ с сохранением величины нормы, т.е. можно ли построить линейных оператор $A$ такой что:
\begin{itemize}
 \item $Ax = A_0x$, если $x\in X_0$
 \item $\Vert A \Vert = \Vert A_0 \Vert$ ?
\end{itemize}


Исследование этого вопроса начинается со случая, когда оператор $A_0$ продолжается с подпространства $X_0$ на подпространство $X_1\subset X$ элементов вида $x+\lambda x_1^0$, где $x\in X_0$, а элемент $x_1^0 \in X_0$ ($A = A_1$, $A_1$ - продолжение оператора $A_0$). Элемент $x_0^1$ назначен, и представление элементов подпространства $X_1$ однозначно: коэффициент $\lambda \in (-\infty , \infty)$ определен однозначно. 

Легко получить необходимое условие этого <<элементарного>> расширения. Для построения элементарного продолжения достаточно априори задать элемент $y_1 = A_1 x_1^0$:
\begin{center}
$A_1(x) = A_!(x_0+\lambda x_1^0) = A_1x_0+\lambda A_1x_1^0 = A_0 x_0 + \lambda y_1$, $x_0 \in X_0$
\end{center}

Пусть оператор $A_1$ существует. Тогда для любого элемента $x_0 \in X_0$ значение
\begin{center}
$\Vert y_1 -A_0x_0 \Vert = \Vert A_1x_1^0 - A_1x_0 \Vert = \Vert A_1(x_1^0 - x_0) \Vert \leqslant \Vert A_1 \Vert \Vert x_1^0 - x_0 \Vert = \Vert A_0 \Vert \Vert x_1^0 - x_0 \Vert_X.$
\end{center}

Рассмотрим шары с центром в точках $A_0x_0$ и радиусами $\Vert A_0 \Vert \Vert x_1^0 - x_0 \Vert$. При любом выборе элемента $x_0 \in X_0$ все сферы $S_{\Vert A_0 \Vert \Vert x_0 - x_1^0 \Vert}(A_0x_0)$ должны содержать заданный общий элемент $y_1$.

Можно доказать, что это свойство пространства $Y$ является достаточным условием продолжения линейных операторов. Такие пространства $Y$ уникальны. Из числа практически интересных пространств такого типа отметим только пространство $M(\mathit{D})$ вещественных ограниченных в конечной области $D$ функций $y(t)$ с нормой $\Vert y\Vert_{M(D)} = \sup \limits_D\lvert y(t) \rvert$ и пространство $R_1$ вещественных чисел.

В общем случае даже элементарные продолжения линейных операторов невозможны.

\section{Продолжение линейных функционалов. Теорема Хана-Банаха.}
В нормированном пространстве $X$ рассмотрим множество $X^*$ линейных функционалов $X^*=\mathcal{L}(X, R_1):$
\begin{center}
$ f\in X^*, f(x)\in R_1$ (пространство $Y=R_1$).
\end{center}
Норма функционала $f$ определяется естественным образом:
$$\lVert f \rVert = \underset{\lVert x \rVert=1}{\sup} \lvert f(x) \rvert,$$
$$\lvert f(x) \rvert \leqslant \lVert f \rVert \cdot \lVert x \rVert_X.$$

\begin{theorem-break}[Об элементарном продолжении]
Пусть $X_0$ подпространство пространства $X$, и на $X_0$ определен функционал $f_0$ с нормой $\lVert f_0 \rVert = \underset{\lVert x \rVert=1, x \in X_0}{sup}|f(x)|$. Пусть элемент $x_1^0 \in X$, но $x_1^0 \notin X_0$.

Введем множество $X_1 \subset X$ элементов вида $x_0+\lambda x_1^0$, где $x_0$ любой элемент пространства $X_0$, $-\infty < \lambda < \infty$. Представление элемента $x_1$ в виде $x_0+\lambda x_1^0$ единственно.
\end{theorem-break}

\begin{proof}
Действительно, если $$x_0+\lambda x_1^0=x_0'+\lambda_1 x_1^0,$$ то $$(x_0-x_0')+(\lambda-\lambda_1)x_1^0=0.$$
Так как $(x_0-x_0') \in X_0$, то $(\lambda-\lambda_1)x_1^0 \in X_0$, что возможно только при $\lambda=\lambda_1$. Тогда $x_0=x_0^1$.

Требуется построить линейный функционал $f_1 \in \mathcal{L}(x_1, R_1)$, такой что
\begin{itemize}
	\item $f_1(x)=f_0(x)$, если $x\in X_0$,
	\item $\lVert f_1 \rVert=\lVert f_0 \rVert$.
\end{itemize}

Функционал $f_1$ построим в виде $f_1=f_0+\lambda f_1(x_1^0)$. Число $f_1(x_1^0)$ выберем позднее.

Приступим к оценкам. Пусть $x_0^1$ и $x_0^2$ два любых элемента множества $X_0$.
$$f_0(x_0^1)-f_0(x_0^2)=f_0(x_0^1-x_0^2) \leqslant \lVert f_0 \rVert \cdot \lVert x_0^1-x_0^2 \rVert = \lVert f_0 \rVert \cdot \lVert (x_0^1+x_1^0)-(x_0^2+x_1^0) \rVert \leqslant$$
$$ \leqslant \lVert f_0 \rVert \cdot (\lVert x_0^1+x_1^0 \rVert + \lVert x_0^2+x_1^0 \rVert) $$
или
$$f_0(x_0^1)-\lVert f_0 \rVert \cdot \lVert x_0^1+x_1^0 \rVert \leqslant f_0(x_0^2)+\lVert f_0 \rVert \cdot \lVert x_0^2+x_1^0 \rVert.$$

Зафиксируем элемент $x_0^2$. Тогда правая часть неравенства зависит только от выбранного элемента $x_0^2$, поэтому левая часть неравенства ограничена сверху: существует
$$\underset{x_0^1 \in X_0}{\sup}\left( f_0(x_0^1)-\lVert f_0 \rVert \cdot \lVert x_0^1+x_1^0 \rVert \right)=\alpha,$$
и при любых элементах $x_0^2$ значение $f_0(x_0^2)+\lVert f_0 \rVert \cdot \lVert x_0^2+x_1^0 \rVert \geqslant \alpha$.

Тогда существует $$\underset{x_0^2 \in X_0}{\inf}\left(f_0(x_0^2)+\lVert f_0 \rVert \cdot \lVert x_0^2+x_1^0 \rVert \right)=\beta, \beta \geqslant  \alpha .$$

В результате получаем
\begin{center}
$f_0(x_0^1)-\lVert f_0 \rVert \cdot \lVert x_0^1+x_1^0 \rVert \leqslant \alpha \leqslant \beta \leqslant f_0(x_0^2)+\lVert f_0 \rVert \cdot \lVert x_0^2+x_1^0 \rVert$, $x_0^1$ и $x_0^2 \in X_0$.
\end{center}

На отрезке $[\alpha ; \beta]$ выберем любое число $\gamma$. Тогда для любого элемента $x_0 \in X_0$:
$$f_0(x_0)-\lVert f_0 \rVert \cdot \lVert x_0+x_1^0 \rVert \leqslant \gamma \leqslant f_0(x_0)+\lVert f_0 \rVert \cdot \lVert x_0+x_1^0 \rVert$$
и
$$
\begin{cases}
    f_0(x_0)-\gamma &\leqslant \lVert f_0 \rVert \cdot \lVert x_0+x_1^0 \rVert\\
    \gamma - f_0(x_0) &\leqslant \lVert f_0 \rVert \cdot \lVert x_0+x_1^0 \rVert
\end{cases}.
$$

Таким образом, верна оценка
\begin{equation}
\label{32_1}
f_0(x_0)-\gamma \leqslant \lVert f_0 \rVert \cdot \lVert x_0+x_1^0 \rVert.
\end{equation}

Для построения значения функционала $f_1$ назначим значение $f_1(x_1^0)=-\gamma$. Значение функционала $f_1$ на $X_1$ равно
$$ f_1(x)=f_1(x_0+ \lambda x_1^0)=f_0(x_0)-\lambda \gamma.$$

Ясно, что функционал $f_1$ аддитивен и однороден. Покажем, что $f_1$ --- линейный функционал.
\begin{align*}
|f_1(x)| &= |f_0(x_0)-\lambda \gamma| \leqslant \left|\lambda f_0\left(\frac{x_0}{\lambda}\right)-\lambda \gamma\right| \leqslant
|\lambda| \left|f_0\left(\frac{x_0}{\lambda}\right)- \gamma\right| \overset{\mbox{(\ref{32_1})}}{\leqslant} |\lambda|\cdot \lVert f_0 \rVert \cdot \left \lVert \frac{x_0}{\lambda} + x_1^0 \right \rVert =\\
&= \lvert \lambda \rvert \cdot \lVert f_0 \rVert \cdot \left \lVert \frac{x_0+\lambda x_1^0}{\lambda} \right \rVert = \lVert f_0 \rVert \cdot  \lVert x_0+\lambda x_1^0 \rVert = \lVert f_0 \rVert \cdot  \lVert x \rVert.
\end{align*}

Следовательно, $f_1 \leqslant f_0$ и функционал $f_1$ линеен.

Обратное неравенство $f_0 \leqslant f_1$ получаем соответственно из оценки:
$$ \lVert f_1 \rVert=\underset{x \in X, \; \lVert x \rVert = 1 }{\sup} |f_1(x)| \geqslant \underset{x \in X_0, \; \lVert x \rVert = 1 }{\sup} |f_1(x)|=\lVert f_0 \rVert.$$

Итак, $\lVert f_1 \rVert=\lVert f_0 \rVert$.

Элементарное продолжение функционала определяется выбором значения $f_1(x)$ на элементе $x_1^0$.
\end{proof}

\begin{theorem}[Хана-Банаха в случае сепарабельного пространства]
В сепарабельном пространстве $X$ существует всюду плотное счетное множество $D$ линейно независимых элементов. В подпространстве $X_0 \subset X$ определен линейный функционал $f_0$ с нормой $\lVert f_0 \rVert$. В множестве $D$ укажем элементы $x_1, x_2, \ldots, x_n,\ldots$, не принадлежащие $X_0$.

Построим подпространство
$$X_1=X_0+\{ \lambda\}x_1 $$
и элементарное расширение $f_1$ функционала $f_0$ на $X_1$, $\lVert f_1 \rVert= \lVert f_0 \rVert$.

Построим подпространство
$$X_2=X_1+\{ \lambda\}x_2 $$
и элементарное расширение $f_2$ функционала $f_1$ на $X_2$, $\lVert f_2 \rVert= \lVert f_1 \rVert$.

Продолжая этот процесс, получим функционал $f$, определенный на $X=X_0+\overbar{\overset{\infty}{\underset{k=1}{\bigcup}}X_k}$, $\lVert f \rVert= \lVert f_0 \rVert.$
\end{theorem}
Теорема Хана-Банаха верна и для произвольного нормированного пространства $X$: 
\begin{theorem}[Хана-Банаха, $\sim$ 1930 г.]
Пусть $X$ вещественное нормированное пространство и функционал $f_0 \in \mathcal{L}(X_0, R_1)$, где $X_0$ --- подпространство пространства $X$. Существует продолжение функционала $f_0$ на все пространство $X$ с сохранением нормы. 
\end{theorem}
\begin{remark}
\hfill \newline
\begin{enumerate}
\item Продолжение функционала $f_0$ не единственно.
\item Если $X$ - комплексное нормированное пространство, то теорема Хана-Банаха верна (значения функционалов $f_0$ и $f_1$ --- комплексные числа).
\end{enumerate}
\end{remark}

\section{Следствия теоремы Хана-Банаха}

\begin{corollary}[1. Теорема о достаточном числе функционалов]
Для любого элемента $x_0 \in X$, $(x_0 \ne0)$ существует функционал $f \in X^*$, такой, что $\lVert f \rVert=1$, $f(x_0)=\lVert x_0\rVert$.
\end{corollary}

\begin{proof}

Обозначим $X_0 \subset X$ множество элементов пространства $X$ вида $\{\lambda\}x_0$. Определим функционал $f_0$ на множестве $X_0$: $f_0(x_0)=\lambda\lVert x_0 \rVert$. Ясно, что на $X_0$ функционал $f_0$ линеен:
\begin{equation*}
| f_0(x)|=| \lambda | \lVert x_0 \rVert=\lVert \lambda x_0 \rVert=\lVert x\rVert; \lVert f_0\rVert=1.
\end{equation*}
При $x=x_0$  $f_0(x_0)=\lVert x_0 \rVert$. Продолжая этот функционал на всё пространство $X$, получим (по теореме Хана-Банаха) функционал $f$:
\begin{equation*}
f(x_0)=f_0(x_0)=\lVert x_0 \rVert ; \lVert f\rVert=1.
\qedhere
\end{equation*}
\end{proof}

\begin{corollary}[2]
Пусть $\Omega$ линейное множество в пространстве $X$ и элемент $x_0 \subset X$ не принадлежит $\Omega$: $\smash{\displaystyle \inf_{x' \in \Omega}} \lVert x_0-x' \rVert = d>0$. Тогда существует функционал $f \in X^*$, такой, что $f(x')=0$, если $x' \in \Omega$ и $\lVert f \rVert = 1$,$f(x_0)=d$
\end{corollary}
\begin{proof}
Образуем элементарное расшириение $X_0$ множества $\Omega$, т.е. множество элементов вида $x=x'+\{ \lambda \}x_0$, где $x' \in \Omega$. Определим функционал $f_0 \subset X^*_0$: $f_0(x)=\lambda d$ при $x \in X_0$. Ясно, что $f_0(x_0)=d$ и $f_0(x)=0$ при $x \in \Omega$. Оценим норму $\lVert x \rVert$:
\begin{equation*}
\lVert x\rVert=\lVert x'+\lambda x_0\rVert=| \lambda | \lVert x_0+\frac{x'}{\lambda}\rVert=|\lambda |\lVert x_0-\frac{x'}{\lambda}\rVert \geqslant | \lambda | \smash{\displaystyle \inf_{x' \in \Omega}}\lVert x_0-x' \rVert=| \lambda |d=|f_0(x)|.
\end{equation*}
Таким образом $|f_0(x)|\leqslant \lVert x \rVert$ и $\lVert f_0 \rVert \leqslant 1$.

Чтобы получить противоположное неравенство $\lVert f_0 \rVert \geqslant 1$, найдем для заданное $\varepsilon >0$ элемент $x'_{\varepsilon}\in \Omega$, такой, что $\lVert x_0-x'_{\varepsilon} \rVert \leqslant d+\varepsilon$. Ясно, что $d=f_0(x_0)=f_0(x_0-x'_{\varepsilon})\leqslant \lVert f_0\rVert\lVert x_0-x'_{\varepsilon} \rVert < \lVert f_0\rVert (d+\varepsilon)$, откуда получаем неравенства $\frac{d}{d+\varepsilon}\leqslant \lVert f_0\rVert \leqslant 1$ и стало быть $\lVert f_0\rVert=1$ ($\varepsilon$ любое $>0$).

По теореме Хана-Банаха существует продолжение $f$ функционала $f_0$ на все пространство $X$:
\begin{equation*}
\lVert f \rVert=1; \mbox{ при }  x \in X_0, \;  f(x)=\lambda d \; (f(x)=0, \; x\in \Omega; \; f(x_0)=d).
\qedhere
\end{equation*}
\end{proof}

\begin{corollary}[3]
Если для всех $f \subset X^*$ значения $f(x)=0$, то $x=\mathbb{0}$.
\end{corollary}

\begin{corollary}[4]
Если для всех функционалов $f \in X^*$, норма которых равна $1$, выполнено неравенство $f(x_0)\leqslant C$, то $\lVert x_0 \rVert \leqslant C$.
\end{corollary}

\begin{corollary}[5]
Пусть $\{ x_1, x_2, x_3, \ldots x_n\}$ система линейно независимых элементов нормированного пространства $X$. Тогда существует система функционалов $\{ f_1, f_2, f_3, \ldots f_n\}$, $f_n \in X^*$ такая, что 
\begin{equation*}
f_k(x_i) = 
\begin{cases}
   0 \mbox{ при } i\neq k\\
   1 \mbox{ при } i=k
\end{cases}
\end{equation*}
\end{corollary}
\begin{proof}
Обозначим линейную оболочку системы элементов $\{ x_2, x_3, \ldots x_n\}$ через $\Omega_1$, $\Omega_1 \subset X$. Элемент $x_1 \notin \Omega_1$: $\rho(x_1, \Omega)=d_1>0$. Согласно следствию (2) существует функционал $f_1 \in X^*$, такой, что 
\begin{equation*}
\lVert f_1 \rVert =d_1, \; f_1(x_1)=1, \; f_1(\Omega_1)=0: \; f_1(x_2)=0, \; f_1(x_3)=0, \;\ldots, \; f_1(x_n)=0.
\end{equation*}
Обозначим $\Omega_2$ линейную оболочку элементов $\{ x_1, x_3, \ldots x_n\}$. Элемент $x_2 \in \Omega_2$: $\rho(x_2, \Omega_2)=d_2$. По следствию (2) существует функционал $f_2 \in X^*$ такой, что 
\begin{equation*}
\lVert f_2 \rVert=d_2 > 0, \; f_2(x_2)=1, \; f_2(\Omega_2)=0: \; f_2(x_1)=0, \; f_2(x_3)=0, \;\ldots, \; f_2(x_n)=0.
\end{equation*}
Продолжая этот процесс, получим функционалы $f_k \in X^*$, $k=1, 2, \ldots, n$
\begin{equation*}
f_k(x_i) = 
\begin{cases}
   0 \mbox{ при } i\neq k\\
   1 \mbox{ при } i=k
\end{cases}
\qedhere
\end{equation*}
\end{proof}
\begin{corollary}[6. Теорема об опорной плоскости]
Пусть $f \in X^*$. Плоскостью (гиперплоскостью) в пространстве $X$ называется множество $L$ элементов, удовлетворяющих условию $f(x)=C$, где $C$ - вещественное число.
\end{corollary}

\begin {definition}
Если для всех элементов множества $E$ верно неравенство $f(x)=C$ (или $f(x) \geqslant C$), то говорят, что $E$ лежит по одну сторону от плоскости $L$. Плоскость $L$ называется опорной для множества $E$ в точке $x_0 \in E$, если $E$ лежит по одну сторону от и $x_0 \in L$.
\end {definition}

\begin{theorem}
Для любого замкнутого шара $\overbar{V}_r(\mathbb{0})$, ($\lVert x \rVert \leqslant r$) и для любой точки $x_0 \in S_r(\mathbb{0})$ существует опорная плоскость $L$ к шару $\overbar{V}_r(\mathbb{0})$ в точке $x_0$.
\end{theorem}

\begin{proof}
Построим подпространство $X_0$ элементов вида $\{ \lambda \}x_0$. На $X_0$ определим функционал $f_0$:
\begin{equation*}
f_0(x)=\lambda \lVert x_0 \rVert = \lambda r, \; |f_0(x)|=|\lambda|\lVert x_0 \rVert=\lVert \lambda x_0 \rVert=\lVert x \rVert; \; \lVert f_0 \rVert=1.
\end{equation*}
Продолжим функционал $f_0$ на все пространство $X$ с сохранением нормы. Опорная плоскость к шару $\overbar{V}_r(\mathbb{0})$ в точке $x_0$ определяется уравнением $f(x)=r$. Действительно, для $x \in \overbar{V}_r(\mathbb{0})$:
\begin{equation*}
f(x)\leqslant 1 \lVert x \rVert \leqslant r,
\end{equation*}
при $x_0 \in S_r(\mathbb{0})$ значение $f(x_0)=r$.
\qedhere
\end{proof}

\chapter{Сопряженное пространство. Сопряженный оператор}

\section{Сопряженное пространство}

Согласно определению пространством $X^*$, сопряженным нормированному пространству $X$, называется пространство линейных функционалов $f$, заданных на всем $X$. Все результаты, полученные для линейных операторов, переносятся на частный случай линейных функционалов.

Пространство $X^*$ --- полное пространство (пространство Банаха), $\lVert f\rVert = \sup\limits_{\lVert x\rVert = 1}\lvert f(x)\rvert$ (глава 2, $\S 2$).

В дальнейшем наряду с записью значения функционала $f$ элементы $x \in X$ мы будем обозначать
\begin{equation*}
f(x) =  \langle x, f \rangle 
\end{equation*}
Такое обозначение имеет своё обоснование.
\begin{theorem}[Рисса об общем виде линейного функционала в пространстве Гильберта $H$]
Для любого $f \in H^*$ существует единственный элемент $y \in H$, такой что $ \langle x, f \rangle =(x,y)$ ($f(x)=(x,y)$, $(x,y)$ --- скалярное произведение в $H$).
\end{theorem}

(Фридьёф Рисс, 1880-1956 г., один из основоположников функционального анализа)

\begin{proof}
Обозначим $L$ подпространство элементов $z$ таких, что значения функционала $f$ равно 0: $ \langle z, f \rangle =0$. Можно считать, что $L\ne H$ в противном случае $f(x)=0$ для любого $x\in H$, $\lVert f\rVert = 0$, $y=\mathbb{0}$.

Ортогональное дополнение подпространства $L$ не пусто. Пусть $x_0\perp L$, тогда и элемент $\lambda x_0\perp L$, и можно считать, что $ \langle x_0, f \rangle =1$.

Пусть $x$ --- любой элемент $H$. На элементе $x- \langle x, f \rangle x_0$ значение функционала $f$ равно:
\begin{equation*}
 \langle x- \langle x, f \rangle x_0, f \rangle  =  \langle x, f \rangle - \langle x, f \rangle  \langle x_0, f \rangle  =  \langle x, f \rangle - \langle x, f \rangle  = 0
\end{equation*}

Следовательно элемент $x- \langle x, f \rangle x_0 \subset L$, a $x_0\perp L$:

\begin{equation*}
(x- \langle x, f \rangle x_0, x_0) = 0; (x, x_0) -  \langle x, f \rangle (x_0, x_0) = 0
\end{equation*}

Тогда $(x, f) = \frac{(x, x_0)}{\lVert x_0\rVert^2}$ и в качестве элемента $y$ можно взять элемент $y=\frac{x}{\lVert x_0\rVert^2}$:$ \langle x, f \rangle  = (x, y)$.

Для нормы функционала $f$: $\lvert  \langle x, f \rangle \rvert \leqslant \lVert y\rVert\lVert x\rVert$. Тогда $\lVert f\rVert \leqslant \lVert y\rVert$. С другой стороны $ \langle y, f \rangle  = (y, y) \leqslant \lVert f\rVert\lVert y\rVert$, $\lVert y\rVert\le\lVert f\rVert$

Объединяя эти неравенства, получаем $\lVert f\rVert = \lVert y\rVert$.

\underbar{Единственность элемента $y$}: предположим, что существует другой элемент $y_1 \in H$ такой, что для любого $x\in H$:
\begin{equation*}
 \langle x, f \rangle  = (x, y) = (x, y_1)
\end{equation*}
Тогда $(x, y-y_1) = 0$ и, взяв элемент $x = y-y_1$, получим $\lVert y-y_1\rVert = 0$, т.е. $y_1 = y$.
\end{proof}
Сходимость последовательности функционалов $\{f_n\}\in X^*$ к функционалу $f_0\in X^*$: \textbf{сильная сходимость}, если $\lVert f_n - f_0\rVert \to 0$ при $n\to\infty$; \textbf{поточечная сходимость} $f_n(x)\to f_0(x)$ при $n\to\infty$ для любых элементов $x\in X$. Верна теорема (Банаха-Штейнгауза, глава 2, \S 3):

Для того, чтобы последовательность $f_n$ сходилась поточечно к линейному функционалу, необходимо и достаточно, чтобы:

\begin{enumerate}
\item Нормы $f_n$ были ограничены в совокупности: $\lVert f_n\rVert \leqslant const$.
\item Существуют пределы числовых последовательностей $f_n(x)$ при $n\to\infty$ для всех элементов $x$, принадлежащих множеству $D$
    всюду плотному в $X$.
\end{enumerate}

Введение сопряженного пространства приводит к новому типу сходимости последовательности $\{x_n\}$ элементов $x_n$ пространства $X$.

\begin{definition}
Последовательность $x_n$ сходится к элементу $x^* \in X$ \textbf{слабо}, если числовые последовательности $f(x_n) \rightarrow f(x^*)$ для любого функционала $f \in X^*$. В этом случае пишут $x_n \rightarrow x^*$ \big($ \langle x_n,f \rangle  \ \rightarrow \  \langle x^*, f \rangle ,\ f \in X^*$\big), при этом сами значения $f(x_n)$ и $f(x^*)$ зависят от выбора функционалов $f$.

Как сильный предел последовательности, так и слабый предел единственны: если $x_n\rightarrow~x^*$ и $x_n \rightarrow \widetilde{x}$, то для любого функционала $f$: $f(x^*-\widetilde{x})=0$. По теореме Хана-Банаха (следствие (3)): $x^*=\widetilde{x}$.
\end{definition}

\begin{lemma}
Если $x_n \rightarrow x_0$, то $\lVert x_n \rVert \leqslant \lVert x_0 \rVert$. (Слабо сходящаяся последовательность элементов ограничена по норме).
\end{lemma}
\begin{proof}
Для всех $f \in X*$ значения $f(x_n) \rightarrow f(x_0)$ при $n \rightarrow \infty$. По теореме о достаточном числе функционалов (Глава 3, \textsection 2) для каждого элемента $x_n$ существует функционал $f_n$, такой что $\lVert f_n \rVert =1$ и $f_n(x_n)=\lVert x_n \rVert$. Тогда $f_n(x_n) \rightarrow f_n(x_0)$. Но $|f_n(x_0)| \leqslant \lVert f_n \rVert \cdot \lVert x_0 \rVert = \lVert x_0 \rVert$. Следовательно, предел $\{ \lVert x_n \rVert \}$ может и не существовать, $\overbar{\lim}\lVert x_n \rVert \leqslant \lVert x_0 \rVert$.
\end{proof}

Понятие слабой сходимости последовательности элементов порождает также понятие \textbf{слабо фундаментальной последовательности} $\{x_n\}$: последовательность $\{x_n\}$ называется слабо фундаментальной, если для каждого функционала $f \in X^*$ последовательность чисел $\{f(x_n)\}$ фундаментальна.

В соответствии с этим определением говорят, что множество $K$ \textbf{слабо компактно}, если из любой последовательности его элементов можно составить ее подпоследовательность слабо фундаментальную: для последовательности $\{x_n\}$ существует ее подпоследовательность $\{x_{N(k)}\}$ такая что числовые последовательности $\{f(x_{N(k)})\}$ имеют предел при $k \rightarrow \infty$. Значения $f(x_{N(k)})$ и предельное значение зависят от выбора функционала $f$.

В конечномерном пространстве любое ограниченное множество компактно. Естественно ожидать, что в бесконечномерных пространствах условия слабой компактности не будут сложными.

\begin{theorem}
В гильбертовом пространстве $H$ любое ограниченное множество слабо компактно.
\end{theorem}

\begin{proof}
Проведем доказательство для сепарабельных гильбертовых пространств. В этих пространства существует не более чем счетный базис элементов $\psi_1, \psi_2, \ldots, \psi_n, \ldots$, $(\psi_i, \psi_j)=0, i\neq j$; ${\lVert \psi_i \rVert}_H=1$.

Обозначим множество элементов пространства $H$, таких что ${\Vert x \Vert} _H \leqslant C$ через $K$. Пусть $\lbrace x_n \rbrace$ - произвольная последовательность элементов множества $K$.

\begin{enumerate}
\item Для всех элементов $x_n$ вычислим значения функционала $f_1$:

\begin{center}
$f_1(x)= \langle x,\Psi_1 \rangle =(x,\Psi_1)$.
\end{center}
Числовая последовательность $f_1(x_n)$ ограничена:
\begin{center}
$\vert f_1(x_n)\vert = \vert \langle x,\Psi_1 \rangle \vert \leqslant \Vert x_n\Vert \leqslant C$.
\end{center}
По теореме Больцано-Вейерштрасса в $\lbrace f_1(x_n) \rbrace$ существует сходящаяся подпоследовательность, обозначим её 
$ \langle x_{N(1,K)},\Psi_1 \rangle $,
\begin{center}
$\lbrace x_{N(1,K)} \rbrace \subset
 \lbrace x_n \rbrace$, $ \langle x_{N(1,K)},\Psi_1 \rangle  \to \alpha_1$ при $K\to\infty$
\end{center}

\item Рассмотрим последовательность $\lbrace x_{N(1,K)} \rbrace$ и вычислим значения функционала $f_2$:
\begin{center}
$f_2(x)= \langle x,\Psi_2 \rangle $ 
\end{center}
на элементах этой последовательности. Числовая последовательность $  \langle  x_{N(1,K)},\Psi_2 \rangle $ ограничена, существует её сходящаяся подпоследовательность. Обозначим её 
$ \langle x_{N(2,K)},\Psi_2 \rangle $:
\begin{center}
$ \langle x_{N(2,K)},\Psi_2 \rangle  \to \alpha_2$ при
 $K\to\infty$, 
 \end{center}
 \begin{center}
$\lbrace x_{N(2,K)} \rbrace \subset 
\lbrace x_{N(1,K)} \rbrace \subset 
 \lbrace x_n \rbrace$,
  \end{center}
 \begin{center}
 $ \langle x_{N(2,K)},\Psi_1 \rangle  \to \alpha_1$ при
 $K\to\infty$. 
\end{center}
\end{enumerate}

Продолжая этот процесс, получим последовательности 
$\lbrace x_{N(i,K)} \rbrace$, $i=1,2,3\ldots$, такие что 
 \begin{center}
 $\lbrace x_{N(1,K)} \rbrace \supset
\lbrace x_{N(2,K)} \rbrace \supset
\lbrace x_{N(3,K)} \rbrace \supset \ldots$
\end{center}
и для которых
\begin{align*}
\langle x_{N(i,K)},\Psi_i \rangle &\to \alpha_i \mbox{ при } K\to\infty,\\
\langle x_{N(i-1,K)},\Psi_{i-1} \rangle &\to \alpha_{i-1} \mbox{ при } K\to\infty,\\
&\cdots\\
\langle x_{N(1,K)},\Psi_{1} \rangle &\to \alpha_{1} \mbox{ при } K\to\infty.
\end{align*}

Рассмотрим <<диагональные элементы>> $ x_{N(i,i)}$. Для них при любом фиксированном $K$ значения $\langle x_{N(i,i)},\Psi_K \rangle $ стремятся к $\alpha_K$ при  $i \to \infty$. Так как
\begin{equation*}
 \sum\limits_{k=1}^n 
{\vert \langle x_{N(i,i)},\Psi_K \rangle  \vert}^2 \leqslant C^2
\end{equation*}
  при любом $i$, то переходя к пределу при $i\to\infty$ получим $\sum\limits_{k=1}^n \alpha_k^2 \leqslant C^2$. Так как $n$ любое, то $\sum\limits_{k=1}^\infty \alpha_k^2 <+\infty$. Следовательно существует элемент 
\begin{equation*}
\widetilde{x}= \sum\limits_{k=1}^\infty 
\alpha_k\Psi_{k} \in K.
\end{equation*}

Покажем, что последовательность $\lbrace x_{N(m,m)} \rbrace$ слабо сходится к элементу $\widetilde{x}$. Рассмотрим произвольный элемент $y$ пространства $H$. Он определяет функционал $f$:
 
\begin{equation*}
f(x)=(x,y)
\end{equation*}
Как элемент пространства $H$ элемент $y$ может быть представлен в виде 
\begin{equation*}
y=\sum\limits_{k=1}^\infty y_k\Psi_k =
\sum\limits_{k=1}^n y_k\Psi_k +
\sum\limits_{k=n+1}^\infty y_k\Psi_k
\end{equation*}
Для заданного $\varepsilon$ найдём номер $n(\varepsilon)$, такой что при $n>n(\varepsilon)$
\begin{equation*}
\sum\limits_{k=n+1}^\infty y_k^2 < {\varepsilon}^2 
\end{equation*}
Зафиксируем значение $n$, $n>n(\varepsilon)$.
Значение функционала $f$ на элементе $x_{N(m,m)}-\widetilde{x}$ равно
\begin{equation*}
 \langle x_{N(m,m)}-\widetilde{x},y \rangle = 
 \langle x_{N(m,m)}-\widetilde{x},\sum\limits_{k=1}^n y_k\Psi_k \rangle +
 \langle x_{N(m,m)}-\widetilde{x},\sum\limits_{k=n+1}^\infty y_k\Psi_k \rangle 
\end{equation*}
Во втором слагаемом значение
\begin{equation*}
\vert  \langle x_{N(m,m)},\sum\limits_{k=n+1}^\infty y_k\Psi_k \rangle  \vert \leqslant
C\cdot\varepsilon
\end{equation*}
и значение 
\begin{equation*}
\vert  \langle \widetilde{x},\sum\limits_{k=n+1}^\infty y_k\Psi_k \rangle  \vert \leqslant
C\cdot\varepsilon
\end{equation*}
В первом слагаемом значение
\begin{equation*}
\vert  \langle x_{N(m,m)}-\widetilde{x},
\sum\limits_{k=1}^n y_k\Psi_k \rangle  \vert \leqslant
{ \Big( \sum\limits_{k=1}^n{[ \langle x_{N(m,m)},\Psi_k \rangle  - \alpha_k]}^2 \Big) }^{\frac{1}{2}}\cdot
 \lVert f\lVert
\end{equation*}
Так как значение $n$ фиксировано, а значения
$$\langle x_{N(m,m)},\Psi_K \rangle  \to \alpha_K$$
при
$$m\to\infty,$$
то для достаточно больших $m$:
\begin{equation*}
\sum\limits_{k=1}^n{[ \langle x_{N(m,m)},\Psi_k \rangle  - \alpha_k]}^2 < \varepsilon^2.
\end{equation*}
Таким образом
\begin{equation*}
\vert f(x_{N(m,m)}-\widetilde{x})\vert=
\vert  \langle x_{N(m,m)}-\widetilde{x},f \rangle  \vert =
\vert (x_{N(m,m)}-\widetilde{x},y) \vert \leqslant
\varepsilon [\lVert f\lVert + 2\varepsilon]
\end{equation*}
для любого $f \in H^*$, если
 $\lbrace x_n\rbrace \in K$ при достаточно больших $m$, т.е. 
\begin{equation*}
x_{N(m,m)} \to \widetilde{x}
\end{equation*}
на множестве $K$, и множество $K$ слабо компактно в сепарабельном гильбертовом пространстве.
\end{proof}

В доказательстве существенную роль играет теорема Рисса: $H=H^*$ и далее $(H^*)^*=H$. Банахово пространство $X$, для которого ${(X^*)}^*=X$ называется рефлексивным. В случае $X=L_p(T)$ можно показать, что общий вид линейного функционала определяется элементами $y \in L_q(T)$, 
$\frac{1}{p}+ \frac{1}{q}=1$:

\begin{equation*}
f(x)= \langle x,y \rangle =\displaystyle\int\limits_T x(t)y(t)dt
\end{equation*}
(при $p=1$ пространство $L_\infty(T)$ - пространство измеримых и почти везде конечных\\ функций).\\\\
Ясно, что 
\begin{equation*}
{(L_p^*(T))}^*=L_p(T)
\end{equation*}
и пространство $L_p(T)$ рефлексивно.
\begin{theorem}
Верна общая теорема: условие ${\lVert x \lVert}_X \leqslant C$ для элементов множества $K \subset X$ является необходимым и достаточным условием слабой компактности множества $K$ в рефлексивных пространствах $X$.
\end{theorem}
В частности множество $K \in L_p(T)$ элементов, таких что 
\begin{equation*}
{\lVert x \lVert}_{L_p(T)} \leqslant C
\end{equation*}
слабо компактно.

\section{Сопряженный оператор}
Пусть оператор $A \subset \mathcal{L}(X,Y)$, функционал $f \in Y^*$. Вычислим значения $ \langle Ax,f \rangle $. Эти значения можно трактовать как значения функционала $\varphi$, определённого на элементах пространства $X$:
\begin{equation}
 \langle Ax,f \rangle =\varphi(x)
\end{equation}
Ясно, что функционал $\varphi$ задан на всём пространстве $X$ и зависит от выбора функционала $f$:
\begin{enumerate}
\item функционал $\varphi$ аддитивен и однороден:
\begin{align*}
\varphi(\lambda_1x_1+\lambda_2x_2) &= 
 \langle A(\lambda_1x_1+\lambda_2x_2),f \rangle =
\lambda_1 \langle Ax_1,f \rangle  + \lambda_2 \langle Ax_2,f \rangle  =\\
&= \lambda_1\varphi(x_1) + \lambda_2\varphi(x_2)
\end{align*}
\item функционал $\varphi$ линейный:
$$\lVert \varphi \lVert \leqslant \underset{\lVert x\lVert =1}{\sup} 
\vert \varphi(x) \vert \leqslant  \underset{\lVert x\lVert =1}{\sup} 
\lVert Ax \lVert \cdot \lVert f \lVert =
 \lVert A \lVert \cdot \lVert f \lVert.$$
\end{enumerate}

Таким образом равенство $ \langle Ax,f \rangle  =  \langle x,\varphi \rangle $ определяет линейный функционал $\varphi \in X^*$: каждому $f\in Y^*$ сопоставляется линейный функционал $\varphi\in X^*$, зависящий от выбора оператора $A$: эту зависимость обозначим $\varphi=A^* f$.
Эта зависимость от $A$ однородна и аддитивна:
если $A=\alpha_1A_1+\alpha_2A_2$, то 

$$\varphi = \langle \alpha_1A_1 + \alpha_2A_2 \rangle  = \alpha_1 \langle A_1,f \rangle + \alpha_2 \langle A_2,f \rangle  = \alpha_1A_1^*f + \alpha_2A_2^* f.$$
Оператор $A^*$, определённый на $Y^*$, линеен: согласно неравенству

$$\lVert \varphi \lVert = \lVert A^* f \lVert \leqslant \lVert A \lVert \cdot \lVert f \lVert,$$

$${\lVert A^* \lVert}_{Y^* \to X^*} \leqslant 
{\lVert A \lVert}_{X \to Y} \mbox{ и } A^* \in \mathcal{L}(Y^*,X^*).$$
Докажем, что $\lVert A^* \lVert = 
\lVert A \lVert$. Возьмём произвольный элемент $x_0 \in X$ и вычислим $y_0=Ax_0$. Для элемента $y_0$ по теореме Хана-Банаха существует линейный функционал $f$, такой что $f(y_0)= \lVert y_0 \lVert$, $\lVert f \lVert =1$. Тогда
$$\lVert Ax_0 \lVert = \vert  \langle Ax_0,f \rangle  \vert =
\vert  \langle x_0,A^*f \rangle  \vert \leqslant 
\lVert x_0 \lVert \cdot \lVert A^* \lVert
\cdot \lVert f \lVert = \lVert A^* \lVert
\cdot \lVert x_0 \lVert.$$
Следовательно $\lVert A \lVert \leqslant
\lVert A^* \lVert$ и вместе с оценкой $\lVert A^* \lVert \leqslant 
\lVert A \lVert$ получаем $\lVert A^* \lVert = 
\lVert A \lVert$.
\begin{definition}
Линейный оператор $A^* \in \mathcal{L}(Y^*,X^*)$ называется оператором, сопряжённым с оператором $A \in \mathcal{L}(X,Y)$, если для любого $f \in Y^*$ выполнено:
$$\langle Ax,f \rangle  =  \langle x,A^*f \rangle.$$
\end{definition}
\begin{remark}
Ясно, что $\lVert A^* \lVert = \lVert A \lVert$.
\end{remark}

\begin{example}
Рассмотрим комплексно-значные векторные пространства $X=V_n$, $Y=V_n$. Линейный оператор $A$ задаётся матрицей $\lbrace a_{ij} \rbrace_{i,j=1}^n$ с комплексными элементами $a_{ij}$. Скалярное произведение векторов $x(x_1,x_2,\ldots,x_n)$ и $y(y_1,y_2,\ldots,y_n)$ равно $(x,y) = \sum\limits_{i=1}^n x_i \bar{y}_i = \overbar{(y,x)}$

Координаты вектора $y=Ax$ равны $y_i = \sum\limits_{j=1}^n a_{ij} x_j$. Общий вид функционала в $f$ в $V_n$ определяется указанием элемента $y$:
$f(x) \langle x,f \rangle = (x,y)$
Тогда

\begin{align*}
\langle Ax,y \rangle = (Ax,y) &= \sum\limits_{i=1}^n (Ax)_i \bar{y}_i = \sum\limits_{i=1}^n (\sum\limits_{j=1}^n a_{ij}x_j)\bar{y}_i =\\
&= \sum\limits_{j=1}^n x_j \sum\limits_{i=1}^n a_{ij}\bar{y}_i = \sum\limits_{i=1}^n x_i \sum\limits_{j=1}^n a_{ji}\bar{y}_j =\\
&= \sum\limits_{i=1}^n x_i \sum\limits_{j=1}^n \overbar{(\bar{a}_{ij},y_j)} = \sum\limits_{i=1}^n x_i \overbar{\sum\limits_{j=1}^n(\bar{a}_{ij} ,y_j)} =\\
&= (x,A^{*}y)=\langle x, A^{*}y \rangle
\end{align*} 
Сравнивая полученное равенство $\langle Ax,y \rangle = \langle x, A^{*}y \rangle$ (согласно определению сопряженного оператора) получаем, что
$(A^{*}y)_i = \sum\limits_{j=1}^n \bar{a_{ji}} y_j$.
Элементы матрицы $A^{*}$ получены из комплексно-сопряженной матрицы $\bar{A}$ с последующим транспонированием: $A^{*}=\{\bar{a}_{ji}\}_{j,i=1}^n$
\end{example}

\begin{example}
$X=L_p(a,b)$, $Y=L_q(a,b)$. Оператор $K$ --- интегральный оператор: $y=Kx$, $y(t)=\int\limits_a^b K(t,\tau)x(\tau)d\tau \in L_q(a,b)$.

Значение функционала
$$\langle Kx,f \rangle = L_q(a,b)=\int\limits_a^b (Kx)(t)f(t)dt=$$ 
(где $f \in L_p(a,b)$)
$$= \int\limits_a^b \Big( \int\limits_a^b K(t,\tau)x(\tau)d\tau \Big) f(t)dt = \int\limits_a^b x(\tau) \int\limits_a^b K(t,\tau)f(t)dt d\tau =\\= \int\limits_a^b x(t) \int\limits_a^b K(\tau,t)f(\tau)d\tau dt =$$
(общий вид функционала в $L_p(a,b)$)
$$= \langle x,K^{*}f \rangle$$
и
$$K^{*}f=\int\limits_a^b K(\tau, t)f(\tau)d\tau.$$

Таким образом оператор $K^{*}$ есть интегральный оператор(но из $L_p$ в $L_q$), ядро которого $K^{*}(t,\tau)=K(\tau,t)$.

$$\Vert K \Vert = \Big( \displaystyle\int\limits_a^b \lvert K(t,\tau) \rvert^q d\tau dt  \Big)^{\frac{1}{q}}$$

$$\Vert K^{*} \Vert =  \Big( \displaystyle\int\limits_a^b \lvert K(\tau, t) \rvert^q dt d\tau  \Big)^{\frac{1}{q}} =  \Big( \displaystyle\int\limits_a^b \lvert K(t,\tau) \rvert^q d\tau dt  \Big)^{\frac{1}{q}}$$
\end{example}

\begin{example}
Рассмотри комплексно-значные пространства $X=L_p(a,b)$ и $Y=L_q(a,b)$. Пусть $f$ функционал(комплексно-значный) $f$: $L_p(a,b) \to \mathbb{C}$, где $\mathbb{C}$ - множество комплексных чисел. Общий вид линейного функционала в этом случае определяется заданием элемента $y \in L_q(a,b)$:

$$ f(x) = \langle x, f \rangle = (x,y) = \displaystyle\int\limits_a^b x(t)\overbar{y(t)}dt $$

$$ (\frac{1}{p} + \frac{1}{q} = 1) $$

$$ \langle x, f \rangle  = \bar{(y,x)}  = \displaystyle\int\limits_a^b \overbar{(\bar{x}(t)y(t))} dt = \overbar{\displaystyle\int\limits_a^{-b} (y(t)\bar{x}(t))dt} $$

Величины $\int\limits_a^b y(t)\overbar{x(t)} dt$ будем рассматривать как значение функционала из $L_q \to \mathbb{C}$. Это значение полностью определено заданием элемента $x \in L_p(a,b)$. Этот функционал обозначим $f^{*}(y)$:
$\langle x,f, \rangle = \langle y,\bar{f^{*}} \rangle = \bar{f^{*}}(y)$, $f=\bar{f^{*}}$ и $f^{*}=\bar{f}$,

Таким образом в пространстве $L_q(a,b)$ определен функционал $f^{*}$ и $f^{*}=f$.
\end{example}

\chapter{Теория Рисса линейных уравнений второго рода}

В этой главе мы будем рассматривать вполне непрерывные операторы.

\section{Теорема Шаудера}

\begin{definition}
Последовательность $\{y_n\}$ элементов пространства $Y$ называется \textbf{компактной}, если в ней существует фундаментальная подпоследовательность.
\end{definition}

\begin{lemma}[I]
Пусть $Y$ --- банахово пространство. Если последовательность элементов $\{y_n\}$ слабо сходится к элементу $y_0 \in Y$ и компактна, то $y_n\to y_0$ сильно, т.е. \mbox{$\lVert y_n - y_0\rVert_Y\to 0$} при $n\to\infty$.
\end{lemma}

\begin{proof}
(От противного) Предположим, что $\{y_n\}$ не стремится к $y_0$, т.е. существует подпоследовательность $\{{y_n}_k\}$ такая, что $\lVert {y_n}_k-y_0\rVert > \varepsilon$ при достаточно больших значениях $k$. Тогда (по теореме Хана-Банаха глава 3, \S 2, следствие 4) существует функционал $\varphi\in Y^*$, $\lVert\varphi\rVert=1$ такой, что $\varphi({y_n}_k-y_0)=\lVert {y_n}_k-y_0\rVert > \varepsilon$ при всех $k > k_0$. Следовательно последовательность $\{y_n\}$ не имеет слабого предела.
\end{proof}

\begin{lemma}[II]
Пусть $A\subset\sigma(X,Y)$. Если $\{x_n\}\to x_0$, то $Ax_n\to Ax_0$ сильно.
\end{lemma}

\begin{proof}
Так как $\{x_n\}\to x_0$, то $\{\lVert x_n\rVert\}$ ограничена (глава 4, \S 1). Из полной непрерывности оператора $A$ следует, что последовательность элементов $y_n = Ax_n$ \underbar{компактна}.

Покажем, что $Ax_n\to Ax_0$.

Для любого линейного функционала $\varphi\in Y^*$ значения $ \langle A(x_n-x_0), \varphi \rangle  =  \langle x_n-x_0, A^*\varphi \rangle $. Обозначим $A^*\varphi = f\in X^*$:
\begin{equation*}
 \langle A(x_n-x_0), \varphi \rangle  =  \langle x_n-x_0, f \rangle 
\end{equation*}
и так как $x_n\to x_0$, то $ \langle A(x_n-x_0), \varphi \rangle \to 0$ при $n\to\infty$. Тогда \underbar{$Ax_n\to Ax_0$}. По лемме I $\lVert Ax_n-Ax_0\rVert_Y\to 0$ при при $n\to\infty$.
\end{proof}

\begin{theorem}[Шаудер]
Пусть $A\subset\mathcal{\mathcal{L}}(X, Y)$, где $Y$ --- банахово пространство. Тогда операторы $A$ и $A^*$ вполне непрерывны одновременно.
\end{theorem}

\begin{proof}
Пусть $A\subset\sigma(X, Y)$. Рассмотрим последовательность линейных функционалов $\varphi_n\in Y^*$ с нормами $\lVert \varphi_n\rVert = 1$. Покажем, что в последовательности функционалов $\{A^*\varphi_n\}\in X^*$ существует фундаментальная подпоследовательность, что и будет означать полную непрерывность оператора $A^*$.

Обозначим $\{\varphi_n\} = \{y_n\}\in Y^*$ и последовательность функционалов $A^*\varphi_n = A^*y_n = f_n\in X^*$. Ясно, что $\lVert f_n\rVert = \lVert A^*y_n\rVert = \lVert A^*\varphi_n\rVert \leqslant \lVert A^*\rVert\lVert \varphi_n\rVert = \lVert A\rVert$.

Таким образом $\{f_n\}$ ограничена в совокупности. Функционалы $f$ зависят от выбранного $y$: $f_n = f_n(y) = f(y)$.

Ясно, что если $y''$ и $y'\in Y^*$, то
\begin{equation*}
\lVert f(y'')-f(y')\rVert = \lVert A^*y''-A^*y'\rVert\le\lVert A\rVert\lVert y''-y'\rVert\le\varepsilon\mbox{, если} \lVert y''-y'\rVert < \delta\mbox{ и }\lVert A\rVert\delta < \varepsilon
\end{equation*}
Таким образом функции $f(y)$ равностепенно непрерывны. Следуя доказательству теоремы Арцела-Асколи (глава I, \S 1) получаем существование фундаментальной подпоследовательности ${f_n}_k = A^*{\varphi_n}_k$ последовательности $A^*\varphi_n$, $\varphi_n\in S_1\subset Y^*$: $A^*$ --- вполне непрерывный оператор.

Если же $A^*\in\sigma(X^*,Y^*)$, то так как $(A^*)^* = A$, то получаем, что и оператор $A$ вполне непрерывен.
\end{proof}

\section{Уравнения второго рода с вполне непрерывными операторами.}
Уравнения второго рода относительно элемента $x \in X$ имеют вид $x-Ax=y$, $A \in \sigma(X,X)$, $X$ -банахово пространство

Уравнения первого рода: $Ax=y$.

\begin{theorem}
Если $A$ вполне непрерывный оператор, то множества $R(E-A)=(E-A)X$ и $R(I-A^*)=(I-A^*)X^*$ замкнуты.
\end{theorem}
\begin{proof}
Пусть послдеовательность $y_n \rightarrow y_0$, $y_n \in R(E-A)$, т.е. $y_n=x_n-Ax_n$, $x_n \in X$. Следует доказать, что существует элемент $x_0 \in X$ такой, что $y_0=(E-A)x_0$.
\begin{enumerate}

	\item Предположим, что нормы элементов $x_n$ ограничены:

Тогда $y_n=x_n-Ax_n$ и $x_n=y_n+Ax_n$. Так как последовательность $Ax_n$ принадлежит компактному множеству, то существует фундаментальная подпоследовательность $Ax_{n_k}$, и мы получаем, что $x_{n_k}=y_{n_k}+Ax_{n_k}$. Так как $y_{n_k} \rightarrow y_0$, то $\{y_{n_k}\}$ - фундаментальная последовательность. Следовательно, существует $\smash{\displaystyle\lim_{k \rightarrow \infty}} x_{n_k} \in X$, который мы обозначим $x_0$. Тогда $Ax_{n_k} \rightarrow Ax_0$. Переходя к пределу в равенствах $x_{n_k}=y_{n_k}+Ax_{n_k}$ при $k \rightarrow \infty$, получим $y_0=x_0-Ax_0$.
	
	\item Остается рассмотреть случай неограниченной последовательности норм $\lVert x_n \rVert$. В этом случае существует последовательность элементов $x_{n_k}$, нормы которых $\lVert x_n \rVert \rightarrow \infty$ при $k \rightarrow \infty$.
	
	Обозначим через $N$ подпространство пространства $X$ элементов $z$, таких, что $(E-A)z=\mathbb{0}$. Ясно, что $(E-A)(x_n-z)=(E-A)x_n=y_n$.
	
	Если в $N$ существует такой элемент $z$, что нормы $\lVert x_n-z \rVert$ ограничены, то согласно пункту $1.$ существует такой элемент $x_0-z \in X$ такой, что $(x_n-z)\rightarrow (x_0-z)$ при $n \rightarrow \infty$ и $y_0=(x_0-z)-A(x_0-z)=x_0-Ax_0 \in R(E-A)$.
	
	Если же для всех элементов $z$ нормы $\lVert x_n-z \rVert$ неограничены, то можно считать, что величины $\alpha_n=\smash{\displaystyle\inf_{z \in N}} \lVert x_n-z \rVert$ стремятся к $\infty$ при $n \rightarrow \infty$.
	
	По определению $\inf$ существуют элементы $z_n \in N$ такие, что 
	
\begin{equation}
\label{52_1}
\alpha_n \leqslant \lVert x_n-z \rVert \leqslant (1+\frac{1}{n})\alpha_n
\end{equation}	
Построим элементы $u_n$: 
$$ u_n = \frac{x_n - z_n}{\lVert x_n - z_n \rVert}, \quad \lVert u_n \rVert = 1.$$
Для этих элементов вычислим элементы $(E-A)u_n$:
\begin{equation*}
(E-A)u_n=\frac{1}{\lVert x_n-z_n \rVert}y_n.
\end{equation*}	
Так как $y_n \rightarrow y_0$, то нормы $\lVert y_n \rVert$ ограничены, поэтому последовательность $(E-A)u_n \rightarrow \mathbb{0}$. Сами же нормы $\lVert u_n \rVert=1$. Согласно пункту $1.$ существует подпоследовательность последовательности $\{u_n\}$, такая, что $u_n \rightarrow u_0 \in X$. В нашем случае $u_0 \in N$, $\lVert u_0 \rVert=1$. Этот случай невозможен. Действительно, рассмотрим величины $\lVert u_n-u_0 \rVert(1+ \frac{1}{n})\alpha_n$. Согласно (\ref{52_1}):

\begin{align*}
\lVert u_n-u_0 \rVert \frac{n+1}{n} \alpha_n &\geqslant \lVert u_n-u_0 \rVert\lVert x_n-z_n \rVert = \lVert \lVert x_n-z_n \rVert(u_n-u_0) \rVert=\\
&=\lVert u_n\lVert x_n-z_n \rVert-u_0\lVert x_n-z_n \rVert \rVert=\lVert x_n-z_n-\lVert x_n-z_n \rVert u_0 \rVert =\\
&= \lVert x_n-(z_n+\lVert x_n-z_n \rVert u_0) \rVert.
\end{align*}

Элемент $z_n+\lVert x_n-z_n \rVert u_0 \in N$ как линейная комбинация элементов $z_n \in N$ и $u_0 \in N$. Поэтому $\lVert x_n-(z_n+\lVert x_n-z_n \rVert u_0) \rVert \geqslant \alpha_n$.

Тогда $\lVert u_n-u_0 \rVert \frac{n+1}{n} \geqslant 1$ и $\lVert u_n-u_0 \rVert \geqslant\frac{n}{n+1}$, что противоречит $u_0= \smash{\displaystyle\lim_{n \rightarrow \infty}}u_n$
\qedhere
\end{enumerate}
\end{proof}
	
	Из этой теоремы следует, что если рассмотреть уравнения 
	\begin{equation*}
	x-Ax=y_n, \mbox{ где } y_n \rightarrow y_0 
	\end{equation*}	
их <<приближенные>> решения $x_n$: $x_n-Ax_n=y_n$, то существует сходящаяся подпоследовательность $\{x_{n_k}\}$, предел которой $x_0 \in X$ при $k \rightarrow \infty$, удовлетворяет уравнению $x_0-Ax_0=y_0$. Элемент $x_0$ может быть и не единственен: он зависит от выбора сходящейся подпоследовательности последовательности $\{Ax_n\}$.

\section{Теоремы Фредгольма}
Пусть $A$ вполне непрерывный оператор в пространстве Банаха $X$: $A \in \sigma(X,X)$. Рассмотрим уравнения второго рода с вполне непрерывными операторами $A$ и $A^*$:
\begin{align}
(E - A)x &= y , \; y \in X \label{fred_1}\\
(E - A)z &= \mathbb{0}, \; z \in N(E-A) \label{fred_2}\\
(I - A^*)f &= \omega , \; f \in X^*, \; \omega \in X^* \label{fred_3}\\
(I - A^*)\psi &= \mathbb{0}, \; \psi \in N(I - A^*) \label{fred_4}
\end{align}
\begin{theorem}[Первая теорема Фредгольма]
Следующие 4 утверждения эквивалентны:
\begin{equation*}
\begin{cases}
\mbox{ 1. Уравнение (\ref{fred_1}) имеет решение при любой правой части } y\\
\mbox{ 2. Уравнение (\ref{fred_2}) имеет только тривиальное решение } z = \mathbb{0}, \; N(E - A) = \lbrace \mathbb{0} \rbrace\\
\mbox{ 3. Уравнение (\ref{fred_3}) имеет решение при любой правой части } \omega \\
\mbox{ 4. Уравнение (\ref{fred_4}) имеет только тривиальное решение } \psi = \mathbb{0}, \; N(I - A^*) = \lbrace \mathbb{0} \rbrace
\end{cases}
\end{equation*}

\end{theorem}
\begin{proof}
Докажем, например, что из 4 следует 1.\\
Пусть выполнено 4: $(I - A^*)=\mathbb{0}$. Предположим противное : 1 не верно: $R(E - A)\neq X$. Пусть $y_0 \in X$, но $y_0 \not \in R(E - A)$. По теореме Хана-Банаха (следствие (2)) существует линейный функционал $f_0 \in X^*$ такой, что $ \langle y_0,f_0 \rangle =1$, $ \langle y,f_0 \rangle =0$ для всех $y \in R(E-A)$. Тогда $ \langle (E-A)x,f_0 \rangle =0$ для всех $x \in X$, $ \langle x,(I-A^*f_0) \rangle =0$ для всех $x \in X$.\\
Тогда $(I-A^*)f_0=\mathbb{0}$, т.е. $f_0 \in N(I-A^*)$ и $f_0 \neq \mathbb{0}$, что противоречит 4.
\end{proof}
\begin{theorem}[Вторая теорема Фредгольма]
Уравнения (\ref{fred_2}) и (\ref{fred_4}) имеют одинаковое конечное число линейно независимых решений.
\end{theorem}
\begin{theorem}[Третья теорема Фредгольма]
Для того, чтобы уравнение (\ref{fred_1}) имело решение, необходимо и достаточно, чтобы $ \langle y,\psi \rangle =0$ для любого решения $\psi$ уравнения (\ref{fred_4}).
\end{theorem}
\begin{proof}
\underline{Необходимость}. Если $N(E-A)={\mathbb{0}}$, то по второй теореме Фредгольма $N(I-A^*)={\mathbb{0}}$. Если же $N(E-A)\neq {\mathbb{0}}$, то уравнение $(E-A)x=y_0$, $y_0 \neq \mathbb{0}$ имеет решение $x_0$. Пусть $\psi \in N(I-A^*)$. Тогда
\begin{center}
$ \langle y_0,\psi \rangle = \langle (E-A)x_0,\psi \rangle = \langle x_0,(I-A^*)\psi \rangle =0$
\end{center}
\underline{Достаточность}. Пусть $ \langle y_0,\psi \rangle =0$ для всех $\psi \in N(I-A^*)$. Предположим, что \\$y_0 \not \in R(E-A)$, т.е. решение уравнения $(E-A)x=y_0$ не существует. Так как по теореме Шаундера $R(E-A)$ есть подпространство, то существует линейный функционал $f\in X^*$, такой что 
\begin{center}
$ \langle y_0,f \rangle =1$ и $ \langle (E-A)x,f \rangle =0$
\end{center}
для любых элементов x,
\begin{center}
$ \langle x,(I-A^*)f \rangle =0$ и $(I-A^*)f=\mathbb{0}$, $f \in N(I-A^*)$,
\end{center}
т.е. $f$ удовлетворяет уравнению (\ref{fred_4}). Таким образом,
\begin{center}
 $f \in N(I-A^*)$, $f\neq \mathbb{0}$ и $ \langle y_0,f \rangle =1$.
\end{center}
Полученное противоречие показывает, что наше предположение неверно.
\end{proof}

\end{document}
